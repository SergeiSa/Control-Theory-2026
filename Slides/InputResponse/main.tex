\documentclass{beamer}

\input{settings.tex}

\newcommand{\degree}{^{\circ}}     


\title{Input Response}
\subtitle{Control Theory, Lecture 7}
\author{by Sergei Savin}
\centering
\date{\mydate}



\begin{document}
\maketitle


%\begin{frame}{Content}
%
%\begin{itemize}
%\item Frequency response
%\item Partial-fraction expansion with sine input
%\item Amplitude and phase shift of a steady-state solution
%\item Bode plot
%\end{itemize}
%
%\end{frame}






\begin{frame}{Dirac delta function}
	% \framesubtitle{O}
	\begin{flushleft}
		
		\emph{Unit impulse} (\emph{Dirac delta function}) $\delta(t)$ is defined as:
		
		\begin{align}
			&\int_{-\infty}^{\infty} \delta(t) dt = 1
			\\
			&\int_{-\infty}^{\infty} \delta(t) \phi(t) dt = \phi(0)
			\\
			&\int_{-\infty}^{\infty} \delta(t-\tau) \phi(\tau) d\tau = \phi(t)
		\end{align}
		
		This function can also be represented as:
		
		\begin{align}
			\delta(t) = 
		\begin{cases}
			\infty & \text{if} \ \ t = 0
			\\
			0 & \text{if} \ \  t \neq 0 
		\end{cases}
		\end{align}
		
		
			\end{flushleft}
\end{frame}


\begin{frame}{Impuse response}
	% \framesubtitle{O}
	\begin{flushleft}
		
		Reminder: given zero initial conditions, the forced state response (solution to the ODE in the state-space form) is:
		%
		\begin{equation}
			\bo{x}(t) = \int_{0}^{t} e^{(t - \tau)\bo{A}} \bo{B}  \bo{u}(\tau) d\tau
		\end{equation}
		
		If the input signal $\bo{u}$ is a Dirac delta function, the solution takes the form:
		%
		\begin{align}
			\bo{x}(t) = \int_{0}^{t} e^{(t - \tau)\bo{A}} \bo{B}  \delta(\tau) d\tau  = e^{t\bo{A}} \bo{B}
		\end{align}
		
		The output of the system (given Dirac delta function as the input) is called \emph{impulse response}:
		%
		\begin{align}
			\bo{h}(t) =\bo{C} e^{t\bo{A}} \bo{B}
		\end{align}
		
		
		
	\end{flushleft}
\end{frame}



\begin{frame}{Impuse response - to transfer function}
	% \framesubtitle{O}
	\begin{flushleft}
		
		Laplace transform of the exponential and the matrix exponential:
		%
		\begin{align}
			\mathcal{L}(e^{at}) &= \frac{ 1}{s - a}
			\\
			\mathcal{L}(e^{\bo{A}t}) &= (\bo{I}s - \bo{A})^{-1}
		\end{align}
		
		 We can find transfer function of the system as a Laplace transform of the impulse response:
		%
		\begin{align}
			W(s) =
			 \mathcal{L}(\bo{h}(t)) =
			 \int_{0}^{\infty} \bo{h}(t) e^{-st} dt = \\
			 =
			 \int_{0}^{\infty} \bo{C} e^{t\bo{A}} \bo{B} e^{-st} dt =
			 \bo{C} (\bo{I}s - \bo{A})^{-1} \bo{B}
		\end{align}
		
		
		
	\end{flushleft}
\end{frame}



\begin{frame}{Impuse response - to solution}
	% \framesubtitle{O}
	\begin{flushleft}
		
		Knowing impulse response, we could solve the system for any input $\bo{u}(t)$ via convolution:
		%
		\begin{align}
			\bo{y}(t) = (h * u) (t) = \int_{0}^{\infty} \bo{h}(t - \tau) \bo{u}(\tau) d\tau
		\end{align}
		
		We assume that our signals are zero for $t < 0$, hense the lower limit of the intergal.
		
		
	\end{flushleft}
\end{frame}




\begin{frame}{Impuse response - to step response}
	% \framesubtitle{O}
	\begin{flushleft}
		
		Integral of matrix exponential:
		%
		\begin{align}
			\int_0^t e^{\bo{A}\tau} d \tau  = \bo{A}^{-1} (e^{\bo{A} t} - \bo{I})
		\end{align}
		
		This can be proven by differentiation: $\frac{d}{dt}\left ( \bo{A}^{-1} (e^{\bo{A} t} - \bo{I}) \right) = e^{\bo{A}t} $.
		
		\bigskip
		
		\emph{Step response} is the output of the system, given a step function as input:
		%
		\begin{align*}
			\bo{y}_{step} (t) &=  \int_0^t \bo{h}(t - \tau) 1(\tau) d\tau =  \int_0^t \bo{C} e^{\bo{A}(t - \tau)} \bo{B} d \tau =
			\\
			&= \bo{C} \bo{A}^{-1} (e^{\bo{A} t} - \bo{I}) \bo{B}
		\end{align*}
		
		{\small
		Note that for a SISO system $\bo{y}_{step} (t) = \bo{C} \bo{A}^{-1} (e^{\bo{A} t} - \bo{I}) \bo{B}$ represents a single graph; for a MIMO system it is matrix of graphs, corresponding to each input-output channel pair.}
		
		
	\end{flushleft}
\end{frame}





\begin{frame}{Step response - to steady-state gain}
	% \framesubtitle{O}
	\begin{flushleft}
		
		Steady-state (DC) gain describes how the system amplifies a constant input, after the transient process is over; it can be computed as a step response $\bo{y}_{step} (t) $, as $t \rightarrow \infty$:
		%
		\begin{align}
			\bo{K}_{DC} = 
			\lim\limits_{t\rightarrow \infty} \left( \bo{C} \bo{A}^{-1} (e^{\bo{A} t} - \bo{I}) \bo{B} \right)
			=   -\bo{C} \bo{A}^{-1}  \bo{B}
		\end{align}
		%
		Equivalently:
		%
		\begin{align*}
			\bo{K}_{DC} =  \int_0^\infty \bo{h}(\tau) 1(\tau) d\tau =  - \bo{C} \bo{A}^{-1}  \bo{B}
		\end{align*}
		
		
		We can also compute it as the value of the system's transfer function $W(s) =  \bo{C} (\bo{I}s - \bo{A})^{-1} \bo{B}$, at $s=0$:
		
		\begin{align}
			\bo{K}_{DC} = 
			W(0) =- \bo{C} \bo{A}^{-1}  \bo{B}
		\end{align}
		
		\textcolor{mygray}{More in the appendix}
		
	\end{flushleft}
\end{frame}




\begin{frame}{Exponentials as input}
	\begin{flushleft}
		
		Consider a linear system with an exponential $\bo{u}(t) = \bo{v} e^{st}$ as the input ($\bo{v} = \text{const}$). We can solve it via convolution:
		%
		\begin{align}
			\bo{y}(t) = (h * u) (t) = \int_{0}^{\infty} \bo{h}(\tau) \bo{v} e^{s(t-\tau)}d\tau 
			=
			\\
			=   
			e^{st} \underbrace{ \int_{0}^{\infty} \bo{h}(\tau) e^{-s\tau}d\tau}_{\text{Laplace of } \bo{h} =W(s)} \bo{v}  =  W(s) \bo{v} e^{st}
		\end{align}
		
		So, exponentials are preserved by the LTI dynamics. Repeating the same process as before, but for $\bo{u}(t) = \bo{v} e^{j \omega t}$ we get:
		%
		\begin{align}
			\bo{y}(t) =  W(j \omega) \bo{v} e^{j \omega t}
		\end{align}
		
		where $W(j \omega)$ is a Fourier transform of the impulse response and is equal to the transfer function $W(s)$ evaluated at $s = j \omega$.
		 
		
	\end{flushleft}
\end{frame}


\begin{frame}{Frequency response, 1}
	\begin{flushleft}
		
		Let us consider a scalar case: $y(t) =  W(j \omega) e^{j \omega t}$. We can write $W(j \omega) $ in the polar coordinates:
		%
		\begin{align}
			W(j \omega)  =  |W(j \omega) | e^{j\arg(W(j \omega))}
		\end{align}
		
		Then the signal becomes:
		%
		\begin{align}
			W(j \omega)  =  |W(j \omega) | e^{j(\omega t + \arg(W(j \omega)))}
		\end{align}
		
		
		
	\end{flushleft}
\end{frame}




\begin{frame}{Frequency response, 2}
	\begin{flushleft}
		
		Consider $u(t) = e^{j \omega t} = \cos(\omega t) + j \sin(\omega t)$. Then $\cos(\omega t) = \mathfrak{R}(u(t) )$. We can take advantage of the following fact:
		%
		\begin{align}
			\mathfrak{R}(h * u)(t) = (h * \mathfrak{R}(u))(t)
		\end{align}
		
		to find the response to the harmonic input $\cos(\omega t)$ as the real part of the responce to $u(t) = e^{j \omega t}$:
		%
		\begin{align}
			\bo{y}_h(t) =  \mathfrak{R}(W(j \omega)  e^{j \omega t}) =  
			|W(j \omega) | \cos(\omega t + \arg(W(j \omega)))
		\end{align}
		
		We can see that:
		
		\begin{itemize}
			\item Amplitude amplification is equal to $|W(j \omega) | $;
			\item Phase shift equal to $\arg(W(j \omega))$.
		\end{itemize}
		
	\end{flushleft}
\end{frame}




\begin{frame}{Frequency response, 3}
	\begin{flushleft}
		
		Equivalently, we could derive the same via state-space approach. Consider an input $\bo{u}(t) = e^{j\omega t}$ a particular solution $\bo{x}(t) = \bo{X} e^{j\omega t}$ and its deirvative $\dot{\bo{x}}(t) =j\omega \bo{X} e^{j\omega t}$. Then:
		
		\begin{align}
			j\omega \bo{X} e^{j\omega t} &= \bo{A}\bo{X} e^{j\omega t}+ \bo{B} e^{j\omega t}
			\\
			\bo{X} &= (j\omega\bo{I}-\bo{A})^{-1}\bo{B}
			\\
			\bo{y}(t) &= \bo{C} (j\omega\bo{I}-\bo{A})^{-1}\bo{B} e^{j\omega t}
			\\
			\bo{y}(t) &= W(j \omega) e^{j\omega t}
		\end{align}
		
		
		\textcolor{mygray}{More in the appendix}
		
	\end{flushleft}
\end{frame}



\begin{frame}{Bode plot}
	% \framesubtitle{O}
	\begin{flushleft}
		
		The first key idea of a Bode plot is substitution of purely complex variable $j \omega$ in place of Laplace variable $s$, which can have non-zero real part.
		
		\bigskip
		
		Given a transfer function $W(s)$, $s = \sigma + j \omega$ we can analyse its behaviour when $\sigma = 0$. We can plot:
		
		\begin{itemize}
			\item  its amplitude $a(\omega) = \left| W(j \omega) \right|$,
			
			\item its phase $\varphi(\omega) = \text{atan2}( \text{im}(W(j \omega)), \ \text{real}(W(j \omega))  )$.
		\end{itemize}
		
		
		\bigskip
		
		Bode plot is actually two plots: 
		
		\begin{enumerate}
			\item  $20 \cdot \text{log}(a(\omega))$,
			
			\item $\frac{180}{\pi} \varphi(\omega)$.
		\end{enumerate}
		
		The 20 and log() has to do with the vertical axis being in decibels. 
		
	\end{flushleft}
\end{frame}




\begin{frame}{Bode plot - example}
	% \framesubtitle{O}
	\begin{flushleft}
		
		Consider $W(s) = \frac{1}{1 + s}$. Then $W(j \omega) = \frac{1}{1 + j \omega}$. We can transform it as:
		
		\begin{equation}
			W(j \omega) = \frac{1 - j \omega}{(1 + j \omega)(1 - j \omega)} = 
			\frac{1 - j \omega}{1 + \omega^2}
		\end{equation}
		
		Thus we have $\text{real}(W(j \omega)) = \frac{1}{1 + \omega^2}$ and $\text{im}(W(j \omega)) = - \frac{\omega}{1 + \omega^2}$.
		
		\bigskip
		
		Bode plot is then given as:
		
		\begin{equation}
			a(\omega) = \sqrt{\frac{1 + \omega^2}{(1 + \omega^2)^2}} = 
			\frac{1}{\sqrt{(1 + \omega^2)}}
		\end{equation}
		\begin{equation}
			\varphi(\omega) = \text{atan2} \left(-\frac{\omega}{1 + \omega^2}, \ \frac{1}{1 + \omega^2} \right)
		\end{equation}
		
	\end{flushleft}
\end{frame}





\begin{frame}{Bode plot - stability margins}
	% \framesubtitle{O}
	\begin{flushleft}
		
		Before we discuss the use of Bode plot, let us remember that closed-loop transfer function has form (when simple feedback is used):
		
		\begin{equation}
			W(s) = \frac{G(s)}{1 + G(s)}
		\end{equation}
		
		Substituting $s \longrightarrow j \omega$ we get:
		
		\begin{equation}
			W(\omega) = \frac{G(j \omega)}{1 + G(j \omega)}
		\end{equation}
		
		From this we can see that $W(\omega)$ becomes ill-defined if $G(j \omega) = -1$. Meaning, we want to avoid two things happening simultaneously: the amplitude of $G(j \omega)$ being equal to 1, and its phase (argument) being equal to $180\degree$ (remember, phase of $0\degree$ is pure positive real number, phase of $90\degree$ is pure positive imaginary number, $180\degree$ is pure negative real number, etc.).
		
	\end{flushleft}
\end{frame}





\begin{frame}{Stability margins - example}
	% \framesubtitle{O}
	
	% credit: https://www.electrical4u.com/bode-plot-gain-margin-phase-margin/
	\begin{figure}
		\centering
		\includegraphics[width=0.68\linewidth]{bode-plot}
		\caption{}
		\label{fig:bode-plot}
	\end{figure}
	
	
\end{frame}





\begin{frame}{Bode plot with resonance, 1}
	\begin{flushleft}
		
		Consider a system with resonance:
		%
		\begin{equation}
			\begin{cases}
				\dot{\bo{x}} = 
				\begin{bmatrix}
					0 & 1 \\
					-1 & 0
				\end{bmatrix}
				\bo{x}
				+ 
				\begin{bmatrix}
					0\\
					1
				\end{bmatrix}
				\bo{u} 
				\\
				\bo{y} = 
				\begin{bmatrix}
					1 & 0
				\end{bmatrix}
				\bo{x}
			\end{cases}
		\end{equation}
		
		The state matrix of this system has eigenvalues $\lambda_i = \pm j$ with resonance frequency $\omega = 1$.
		
		\bigskip
		
		The eigenvalues of the closed-loop system $\bo{A} - \bo{B} \bo{C}$ are $\lambda_i = \pm \sqrt{2} j$ with resonance frequency $\omega = \sqrt{2}$.
		
	\end{flushleft}
\end{frame}



\begin{frame}{Bode plot with resonance, 2}
	% \framesubtitle{O}
	
	% credit: https://www.electrical4u.com/bode-plot-gain-margin-phase-margin/
	\begin{figure}
		\centering
		\includegraphics[width=0.95\linewidth]{Resonance}
		\caption{Bode plot of a system with resonance}
		\label{fig:Resonance}
	\end{figure}
	
	
\end{frame}


\begin{frame}{Further reading}
	
	\begin{itemize}
		
		\item K. Ogata Modern Control Engineering (Chapter 7.2)
		
		\item Dorf \& Bishop, Modern Control Systems (Chapter 5.3 Performance of Second-Order Systems; Chapter 8 Frequency Response Methods)		
		
		\item C.T. Chen, Linear System Theory and Design (Chapter 2.3)
		
		\item Nise, N.S. Control systems engineering. John Wiley \& Sons. (Chapter 10 Frequency Response Techniques)	
		
		
		\item \bref{https://control.asu.edu/Classes/MAE318/318Lecture18.pdf}{Matthew M. Peet; Systems Analysis and Control - Lecture 18: The Frequency Response}	
		
		\item \bref{https://youtu.be/_eh1conN6YM}{Control System Lectures - Bode Plots, Introduction}
		
		\item \bref{https://global.oup.com/us/companion.websites/fdscontent/uscompanion/us/static/companion.websites/9780199339136/Appendices/Appendix_F.pdf}{Oxford University Press. s-Domain analysis: poles, zeros, and Bode plots}
		
	\end{itemize}
	
\end{frame}

\myqrframe


\begin{frame}{Steady-State gain: ODE}
	\begin{flushleft}
		
		Given an ODE with a constant input $c = \text{const}$:
		
		\begin{align}
			&a_n y^{(n)} + ... + a_1 \dot y  + a_0 y = b_m u^{(m )}+ ... + b_1 \dot u + b_0 u
			\\
			&u(t) = c 
		\end{align}
		
		This is equivalent to:
		
		\begin{align}
			a_n y^{(n)} + ... + a_1 \dot y  + a_0 y =  b_0 c
		\end{align}
		
		A steady-state solution $y_{ss} = \text{const}$:
		
		\begin{align}
			a_0 y_{ss}  =  b_0 c \\
			y_{ss} = \frac{b_0}{a_0} c
		\end{align}
		
		The quantity $K = \frac{b_0}{a_0}$ is the steady-state gain of the system.
		
	\end{flushleft}
\end{frame}





\begin{frame}{Steady-State gain: Transfer Function}
	%	\framesubtitle{Interesting things done easy}
	\begin{flushleft}
		
		Assume the system  $\mathcal{G}$ represented as a transfer function:
		
		\begin{equation}
			Y(s) = \frac{b_m s^m + ... + b_0}{a_n s^n + ... + a_0} U(s)
		\end{equation}
		
		\bigskip
		
		Then, as any element multiplied by the differential operator $s$ with power higher than 0 is a derivative of $u$ or $y$ and both are 0 at the steady-state solution, the steady-state gain can be found by setting those to zero:
		
		\begin{equation}
			K = \frac{b_0}{a_0}
		\end{equation}		

\end{flushleft}
\end{frame}		



\begin{frame}{Steady-State gain: State Space, 1}
	% \framesubtitle{O}
	\begin{flushleft}
		
		Given an LTI with a constant input $\bo{u}_{ss} = \text{const}$:
		
		\begin{equation}
			\begin{cases}
				\dot{\bo{x}} = \bo{A}\bo{x} + \bo{B}\bo{u}
				\\
				\bo{y} = \bo{C}\bo{x}
			\end{cases}
		\end{equation}
		
		A steady-state solution $\bo{x}_{ss} = \text{const}$:
		
		\begin{equation}
		\begin{cases}
			0 = \bo{A}\bo{x}_{ss} + \bo{B}\bo{u}_{ss}
			\\
			\bo{y}_{ss} = \bo{C}\bo{x}_{ss} 
		\end{cases}
		\end{equation}
		
		\begin{equation}
				\bo{y}_{ss} = -\bo{C} \bo{A}^{-1}\bo{B}\bo{u}_{ss}
		\end{equation}
		
		
		The quantity $K =-\bo{C} \bo{A}^{-1}\bo{B}$ is the steady-state gain of the system.
		
	\end{flushleft}
\end{frame}



\begin{frame}{Steady-State gain: State Space, 2}
	% \framesubtitle{O}
	\begin{flushleft}
		
		For the following LTI:
		
		\begin{equation}
			\begin{cases}
			\begin{bmatrix}
				\dot{x}_1 \\ \dot{x}_2 \\ \dot{x}_3
			\end{bmatrix} 
			=
			\begin{bmatrix}
				0 & 1 & 0 \\ 
				0 & 0 & 1 \\
				-a_0 & -a_1 & -a_2
			\end{bmatrix} 
			\begin{bmatrix}
				x_1 \\ x_2 \\ x_3
			\end{bmatrix} 
			+ 
			\begin{bmatrix}
				0 \\ 0 \\ b_0
			\end{bmatrix}
			\bo{u}
			\\
			\bo{y} = 			
			\begin{bmatrix}
				1 & 0 & 0
			\end{bmatrix}
			\begin{bmatrix}
			x_1 \\ x_2 \\ x_3
			\end{bmatrix} 
		\end{cases}
		\end{equation}
		%
		Then we get:
		%
		\begin{equation}
			\bo{y}_{ss} = 
			-\begin{bmatrix}
				1 & 0 & 0
			\end{bmatrix}
			\begin{bmatrix}
				-a_1 / a_0 & -a_2 / a_0 & -1 / a_0 \\
				1 & 0 & 0 \\
				0 & 1 & 0
			\end{bmatrix}
			\begin{bmatrix}
				0 \\ 0 \\ b_0
			\end{bmatrix}
			\bo{u}_{ss}
		\end{equation}
		
		
		The quantity $K =-\bo{C} \bo{A}^{-1}\bo{B} = \frac{b_0}{a_0}$ is the steady-state gain of the system.
		
	\end{flushleft}
\end{frame}






\begin{frame}{Frequency Response, 1}
	% \framesubtitle{O}
	\begin{flushleft}
		
		Consider LTI with input $\bo{u} = \alpha \cos(\omega t) + \beta \sin (\omega t)$:
		%
		\begin{equation}
			\begin{cases}
				\dot{\bo{x}} = \bo{A}\bo{x} + \bo{B}\bo{u}
				\\
				\bo{y} = \bo{C}\bo{x}
			\end{cases}
		\end{equation}
		%
		with steady-state solution:
		%
		\begin{align}
			x_i &= g_i \cos(\omega t) + h_i \sin (\omega t)
			\\
			\dot x_i &=\omega h_i \cos(\omega t) - \omega g_i \sin (\omega t)
			\\
			y_i &= q \cos(\omega t) + r \sin (\omega t)
		\end{align}
		%
		In the vector form:
		%
		%
		\begin{align}
			\begin{bmatrix} x_1 \\ .. \\ x_n \end{bmatrix} 
			&= 
			\begin{bmatrix} g_1 \\ .. \\ g_n \end{bmatrix} \cos(\omega t) + \begin{bmatrix} h_1 \\ .. \\ h_n \end{bmatrix} \sin (\omega t)
			\\
			\bo{x}
			&= 
			\bo{g} \cos(\omega t) + \bo{h} \sin (\omega t)
		\end{align}
		%
		and:
		%
		$
			\dot{\bo{x}}
			= 
			\omega \bo{h} \cos(\omega t) - \omega \bo{g} \sin (\omega t)
		$
		
		
	\end{flushleft}
\end{frame}





\begin{frame}{Frequency Response, 2}
	% \framesubtitle{O}
	\begin{flushleft}
		
		Thus:
		%
		\begin{align}
			\bo{u} &= \alpha \cos(\omega t) + \beta \sin (\omega t)
			\\
			\bo{x}
			&= 
			\bo{g} \cos(\omega t) + \bo{h} \sin (\omega t)
			\\
			\dot{\bo{x}}
			&= 
			\omega \bo{h} \cos(\omega t) - \omega \bo{g} \sin (\omega t)
		\end{align}
		
		With that we can re-write $\dot{\bo{x}} = \bo{A}\bo{x} + \bo{B}\bo{u}$ as:
		%
		%
		\begin{align*}
			\omega \bo{h} \cos(\omega t) - \omega \bo{g} \sin (\omega t) 
			= \\
			 =\bo{A}\bo{g} \cos(\omega t) +  \bo{A}\bo{h} \sin (\omega t) 
			 + 
			 \bo{B}\alpha \cos(\omega t) + \bo{B} \beta \sin (\omega t)
		\end{align*}
		%
		Grouping terms in front of cosines we get:
		%
		\begin{align}
			\omega \bo{h} &= \bo{A} \bo{g} + \bo{B} \alpha
		\end{align}
		%
		And grouping terms in front of sines :
		%
		\begin{align}
				-\omega \bo{g} = \bo{A} \bo{h} + \bo{B} \beta
		\end{align}
		
		
	\end{flushleft}
\end{frame}



\begin{frame}{Frequency Response, 3}
	% \framesubtitle{O}
	\begin{flushleft}
		
		We study the equation $\bo{y} = \bo{C}\bo{x}$ in the same way: 
		%
		\begin{align}
			q \cos(\omega t) + r \sin (\omega t) 
			= 
			\bo{C}\bo{g} \cos(\omega t) + \bo{C}\bo{h} \sin (\omega t)
		\end{align}
		%
		Grouping terms in front of cosines we get:
		%
		\begin{align}
			q = \bo{C} \bo{g}
		\end{align}
		%
		And grouping terms in front of sines :
		%
		\begin{align}
			r = \bo{C} \bo{h}
		\end{align}
		
		
	\end{flushleft}
\end{frame}



\begin{frame}{Frequency Response, 4}
	% \framesubtitle{O}
	\begin{flushleft}
		
		We have the following equations:
		%
		\begin{align}
			\omega \bo{h} &= \bo{A} \bo{g} + \bo{B} \alpha
		\\
			-\omega \bo{g} &= \bo{A} \bo{h} + \bo{B} \beta
		\\
		q &= \bo{C} \bo{g}
		\\
		r &= \bo{C} \bo{h}
		\end{align}
		
		This can be re-written in a matrix form:
		%
		\begin{align}
			\begin{cases}
				\begin{bmatrix}
					-\bo{A} & \omega \bo{I} \\
					-\omega \bo{I} & -\bo{A}
				\end{bmatrix}
				\begin{bmatrix}
					\bo{g} \\ \bo{h}
				\end{bmatrix}
				=
				\begin{bmatrix}
					\bo{B} & 0 \\  0 & \bo{B}
				\end{bmatrix}
				\begin{bmatrix}
					\alpha \\   \beta
				\end{bmatrix}
				\\
				\\
				\begin{bmatrix}
					q \\  r
				\end{bmatrix}
				= 
				\begin{bmatrix}
					\bo{C} & 0 \\  0 & \bo{C}
				\end{bmatrix}
				\begin{bmatrix}
					\bo{g} \\ \bo{h}
				\end{bmatrix}
			\end{cases}
		\end{align}		
		
		
	\end{flushleft}
\end{frame}




\begin{frame}{Frequency Response, 5}
	% \framesubtitle{O}
	\begin{flushleft}
		
		This system can be written as:
		%
		\begin{align}
			\begin{cases}
				\begin{bmatrix}
					-\bo{A} & \omega \bo{I} \\
					-\omega \bo{I} & -\bo{A}
				\end{bmatrix}
				\begin{bmatrix}
					\bo{g} \\ \bo{h}
				\end{bmatrix}
				=
				\begin{bmatrix}
					\bo{B} & 0 \\  0 & \bo{B}
				\end{bmatrix}
				\begin{bmatrix}
				\alpha \\   \beta
				\end{bmatrix}
				\\
				\\
				\begin{bmatrix}
					q \\  r
				\end{bmatrix}
				= 
				\begin{bmatrix}
					\bo{C} & 0 \\  0 & \bo{C}
				\end{bmatrix}
				\begin{bmatrix}
				\bo{g} \\ \bo{h}
				\end{bmatrix}
			\end{cases}
		\end{align}		
		
		Note that we can compute the steady-state solution for all states of this system:
		%
		\begin{align}
			\begin{bmatrix}
				\bo{g} \\ \bo{h}
			\end{bmatrix}
			= 
				\begin{bmatrix}
					-\bo{A} & \omega \bo{I} \\
					-\omega \bo{I} & -\bo{A}
				\end{bmatrix}^{-1}
				\begin{bmatrix}
					\bo{B} & 0 \\  0 & \bo{B}
				\end{bmatrix}
				\begin{bmatrix}
					\alpha \\   \beta
				\end{bmatrix}
		\end{align}		
		
		We can also solve for the steady-state output:
		%
		\begin{align}
			\begin{bmatrix}
				q \\  r
			\end{bmatrix}
			= 
			\begin{bmatrix}
				\bo{C} & 0 \\  0 & \bo{C}
			\end{bmatrix}
			\begin{bmatrix}
				-\bo{A} & \omega \bo{I} \\
				-\omega \bo{I} & -\bo{A}
			\end{bmatrix}^{-1}
			\begin{bmatrix}
				\bo{B} & 0 \\  0 & \bo{B}
			\end{bmatrix}
			\begin{bmatrix}
				\alpha \\   \beta
			\end{bmatrix}
		\end{align}		
		
		
	\end{flushleft}
\end{frame}






\begin{frame}{Frequency Response, 6}
	% \framesubtitle{O}
	\begin{flushleft}
		
		The map between the input and the output as:
		%
		\begin{align}
			\bo{M}(\omega)
			= 
			\begin{bmatrix}
				\bo{C} & 0 \\  0 & \bo{C}
			\end{bmatrix}
			\begin{bmatrix}
				-\bo{A} & \omega \bo{I} \\
				-\omega \bo{I} & -\bo{A}
			\end{bmatrix}^{-1}
			\begin{bmatrix}
				\bo{B} & 0 \\  0 & \bo{B}
			\end{bmatrix}
		\end{align}		
		
		We can define input coordinates $\zeta = 
		\begin{bmatrix}
			\alpha \\   \beta
		\end{bmatrix}$ and output coordinates $\bo{p} = 
		\begin{bmatrix}
		q \\  r
		\end{bmatrix} = \bo{M}\zeta$. The amptitude amplification can be defined as the ratio:
		%
		\begin{align}
			\text{amp}(\omega)
			= 
			\frac{\sqrt{q^2 + r^2}}{\sqrt{\alpha^2 + \beta^2}}
			=
			\frac{||\bo{p}||}{||\zeta||}
			=
			\frac{||\bo{M}\zeta||}{||\zeta||}
		\end{align}		
		
		Notice that by definition $\underset{\zeta}{\text{max}}\frac{||\bo{M}\zeta||_2}{||\zeta||_2} = ||\bo{M}||_2 = \sigma_{\max}(\bo{M})$, where $\sigma_{\max}(\bo{M})$ is the largest singular value of the matrix.
		
	\end{flushleft}
\end{frame}




\begin{frame}{Frequency Response, 7}
	% \framesubtitle{O}
	\begin{flushleft}
		
		The phase shift can be defined as:
		%
		\begin{align}
			\text{phase}(\omega)
			\sim 
			\angle (\bo{p}) - \angle (\zeta)
			=
			\angle (\bo{M} \zeta) - \angle (\zeta)
		\end{align}		
		
		But matrix $\bo{M}$ is a scaled orthonormal matrix:  
		
		\begin{align}
			\bo{M} = \text{amp}(\omega)
			\begin{bmatrix}
				\cos(\varphi) & -\sin(\varphi) \\
				\sin(\varphi) & \cos(\varphi)
			\end{bmatrix}
		\end{align}				
		%
		Thus:
		%
		\begin{align}
			\text{phase}(\omega)
			\sim 
			\varphi = \text{atan2}(M_{21}, M_{11})
		\end{align}		
		
	\end{flushleft}
\end{frame}



\begin{frame}{MIMO Frequency Response, 1}
	% \framesubtitle{O}
	\begin{flushleft}
		
		Transfer function-based Bode plot relies on a SISO representation of a system. However, choosing input and output one can use it for a MIMO system as well. 
		
		\bigskip
		
		State-space representation naturally points out the connection between inputs and outputs and the resulting response. Consider a fixed input matrix $\bo{B} \in \R^{n \times 1}$; we can ask a question, what choice of the output $\bo{C} \in \R^{1 \times n}$ (assuming $||\bo{C} || = 1$) produces the largest amplitude of the output signal.
		
	\end{flushleft}
\end{frame}




\begin{frame}{MIMO Frequency Response, 2}
	% \framesubtitle{O}
	\begin{flushleft}
		
		From the previous slides we saw that:
		
		\begin{align*}
			 \text{amp}(\omega)
			\begin{bmatrix}
				\cos(\varphi) & -\sin(\varphi) \\
				\sin(\varphi) & \cos(\varphi)
			\end{bmatrix}
			= 
			\begin{bmatrix}
				\bo{C} & 0 \\  0 & \bo{C}
			\end{bmatrix}
			\begin{bmatrix}
				-\bo{A} & \omega \bo{I} \\
				-\omega \bo{I} & -\bo{A}
			\end{bmatrix}^{-1}
			\begin{bmatrix}
				\bo{B} & 0 \\  0 & \bo{B}
			\end{bmatrix}
		\end{align*}		
		
		Defining:
		%
		\begin{align}
			\begin{bmatrix}
				\bo{P} _{11} & \bo{P} _{12} \\
				\bo{P} _{21} & \bo{P} _{22}
			\end{bmatrix}
			= 
			\begin{bmatrix}
				-\bo{A} & \omega \bo{I} \\
				-\omega \bo{I} & -\bo{A}
			\end{bmatrix}^{-1}
			\begin{bmatrix}
				\bo{B} & 0 \\  0 & \bo{B}
			\end{bmatrix},
			\ \ \
			\text{amp}(\omega) = m
		\end{align}		
		
		We find:
		%
		\begin{align}
			\begin{bmatrix}
				m\cos(\varphi) & -m\sin(\varphi) \\
				m\sin(\varphi) & m\cos(\varphi)
			\end{bmatrix}
			= 
			\begin{bmatrix}
			\bo{C}\bo{P} _{11} & \bo{C}\bo{P} _{12} \\
			\bo{C}\bo{P} _{21} & \bo{C}\bo{P} _{22}
			\end{bmatrix}
			\\
			m^2\cos^2(\varphi) + m^2\sin^2(\varphi)
		= 
		\bo{C}\bo{P} _{11} \bo{P} _{11}\T \bo{C}\T + 
		\bo{C}\bo{P} _{21} \bo{P} _{21}\T \bo{C}\T
		\\
		m^2
		= 
		\bo{C}(\bo{P} _{11} \bo{P} _{11}\T +\bo{P} _{21} \bo{P} _{21}\T  )\bo{C}\T
		\end{align}		
		
		
		
	\end{flushleft}
\end{frame}



\begin{frame}{MIMO Frequency Response, 3}
	% \framesubtitle{O}
	\begin{flushleft}
		
		Defining $\bo{N} = \bo{P} _{11} \bo{P} _{11}\T +\bo{P} _{21} \bo{P} _{21}\T$ with decomposition $\bo{N} = \bo{D} \bo{D}\T$ we can find the maximum value of $m$ by maximizing:
		%
		\begin{align*}
			m = \underset{\bo{C}}{\max } \frac{\sqrt{\bo{C} \bo{D} \bo{D}\T \bo{C}\T}}{|| \bo{C}||}
			=
			\underset{\bo{C}}{\max } \frac{||\bo{D}\T \bo{C}\T||}{|| \bo{C}||} = \sigma_{\max}(\bo{D}) =\sqrt{ \sigma_{\max}(\bo{N})}
		\end{align*}		
		
		This allows us:
		
		\begin{itemize}
			\item To find the highest amplification ratio in the system's state-space.
			
			\item To find the output matrix $\bo{C}_{\max}$ which corresponds to this "worse-case scenario"; we find it as the vector in the SVD decomposition of $\bo{N})$ matrix corresponding to the largest singular value.
		\end{itemize}
		
	\end{flushleft}
\end{frame}



\begin{frame}{MIMO Frequency Response, 4}
	% \framesubtitle{O}
	\begin{flushleft}
		
		% TODO: \usepackage{graphicx} required
		\begin{figure}
			\centering
			\includegraphics[width=1 \linewidth]{WorstCaseAmp}
			\caption{Dashed - worst case amplitude response, other two - a particular bode plot}
			\label{fig:worstcaseamp}
		\end{figure}
		
		
	\end{flushleft}
\end{frame}








\end{document}
