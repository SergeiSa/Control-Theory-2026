\documentclass{beamer}

\input{settings.tex}


\title{ODE and State Space}
\subtitle{Control Theory, Lecture 1}
\author{by Sergei Savin}
\centering
\date{\mydate}



\begin{document}
\maketitle





\begin{frame}{Role of Control Theory}
	%\framesubtitle{}
	\begin{flushleft}
		
		% TODO: \usepackage{graphicx} required
		\begin{figure}
			\centering
			\includegraphics[width=1.00\linewidth]{"Control Examples"}

			\label{fig:control-examples}
		\end{figure}
		
		
	\end{flushleft}
\end{frame}


\begin{frame}{Role of this course}
	%\framesubtitle{}
	\begin{flushleft}
		
		We will learn:
		
		\begin{itemize}
			\item how to think about objects that we can control;
			
			\item what goals we can expect to achieve with control;
			
			\item how to control;
			
			\item what problems we face when working with control, and how to deal with them.
		\end{itemize}
		
		\bigskip
		
		This is an introductory course, we aim at fulfilling prerequisites for all advanced courses that require basic control background (Control of Autonomous Vehicles, Nonlinear Control, etc)
		
	\end{flushleft}
\end{frame}



\begin{frame}{Models}
	%\framesubtitle{}
	\begin{flushleft}
		
		Many physical objects that we control can be described using \textbf{ODE}s.
		
		\bigskip 
		
		We begin this course by learning how to write ODEs in a form immediately useful for control purposes, called \textbf{State Space} representation.
		
	\end{flushleft}
\end{frame}



\begin{frame}{Ordinary differential equations, 1st order}
%\framesubtitle{}
\begin{flushleft}

Let us remember the normal form of first-order \emph{ordinary differential equations (ODEs):}

\begin{equation}
    \dot{\bo{x}} = \bo{f} (\bo{x}, t)
\end{equation}

where $\bo{x} = \bo{x}(t)$ is the solution of the equation and $t$ is a free variable (usually - time).

\bigskip

\begin{definition}
We can call an ODE a \emph{dynamical system}, and refer to $\bo{x}$ as the \emph{state} of the dynamical system.  
\end{definition}

\begin{example}
\begin{equation}
    \dot{x} = -2 x
\end{equation}
\end{example}

\end{flushleft}
\end{frame}




\begin{frame}{State}
%	\framesubtitle{1st order}
	\begin{flushleft}
		
		The \emph{state} of a dynamical system is a minimal set of variables that describes the system, in the sense that knowing the current state and all future inputs one can describe the future behavior of the system.
		
		\begin{example}
			For a spring-damper system, the state variables could be position and velocity of the mass.
		\end{example} 
		\begin{example}
			For a double pendulum, the state variables could be joint angles and joint velocities.
		\end{example} 
		
	\end{flushleft}
\end{frame}






\begin{frame}{ODEs, n-th order}
%\framesubtitle{n-th order}
\begin{flushleft}

The normal form of an \emph{n-th order} ordinary differential equation is:

\begin{equation}
	y^{(n)} = f (y^{(n-1)}, y^{(n-2)}, ...\,, \dot{y}, y, t)
\end{equation}
			
where $y = y(t)$ is the solution of the equation. Same as before, it is a \emph{dynamical system}, but this time we need more variables to describe the state of this system, for example we can use the set $\{ y, \ \dot{y} , ...\,,y^{(n-1)} \}$.

\begin{example}[Pendulum]
\begin{equation}
    \ddot{y} = - 0.1 \dot y - 7\sin(y)
\end{equation}
\end{example}


\begin{example}[DC motor under constant voltage]
\begin{equation}
\begin{cases}
    \dot{y}_1 = - 100 \dot{y}_2 -2 y_1  + 10 \\
    \ddot{y}_2 = -0.1 \dot{y}_2 + 100 y_1
\end{cases}
\end{equation}
\end{example}

\end{flushleft}
\end{frame}




\begin{frame}{Linear ODE, 1st order}
\begin{flushleft}

A system of linear ODEs of the first order can be written as:

\begin{equation}
    \dot{\bo{x}} = \bo{A} \bo{x}
\end{equation}

\begin{example}
\begin{equation}
\begin{cases}
    \dot{x}_1 = -20 x_1 + 7 x_2 \\
    \dot{x}_2 = 10 x_1 - 3 x_2
\end{cases}
\end{equation}
\end{example}

\begin{example}
\begin{equation}
\begin{bmatrix}
\dot{x}_1 \\
\dot{x}_2
\end{bmatrix} 
= 
\begin{bmatrix}
-20   & 7  \\
 10 & -3
\end{bmatrix}
\begin{bmatrix}
x_1 \\
x_2
\end{bmatrix} 
\end{equation}
\end{example}

\end{flushleft}
\end{frame}




\begin{frame}{Linear differential equations, n-th order}
%\framesubtitle{n-th order}
\begin{flushleft}

A single linear ODE of the n-th order is often written in the form:

\begin{equation}
    a_n y^{(n)} + 
    ... +
    a_2 \ddot{y} + a_1 \dot{y} + 
    a_0 y = 0
\end{equation}

\begin{example}
\begin{equation}
12 \dddot{y}  + 5 \dot{y} + 
    2 y = 0
\end{equation}
\end{example}

\begin{example}
\begin{equation}
    5 \ddot{y} - \dot{y} + 
    10 y = 0
\end{equation}
\end{example}

\end{flushleft}
\end{frame}




\begin{frame}{ODEs with an input, 1}
	%\framesubtitle{n-th order}
	\begin{flushleft}
		
		Sometimes it is convenient to write an ODE in the form with an \emph{input}, for example:
		
		\begin{equation}
			a_2 \ddot{y} + a_1 \dot{y} + 
			a_0 y = u(t)
		\end{equation}
		
		In this equation $u(t)$ is a function of time. This form offers us many uses:
		
		\begin{itemize}
			\item We can use $u(t)$ to model the \emph{control input}, (e.g. voltage, motor torque) that we directly control.
			
			\item We can use $u(t)$ to model external forces acting on the system.
			
			\item We can substitute particular function instead of $u(t)$, e.g. a sine wave or a step function, to study how the system behaves under such input.
		\end{itemize}
		
	\end{flushleft}
\end{frame}



\begin{frame}{First-order ODEs with an input}
	\begin{flushleft}
		
		Some examples of linear ODEs with one input:
		
		
		\begin{example}
			\begin{equation}
				\begin{cases}
					\dot{y}_1 = -20 y_1 + 7 y_2 + u \\
					\dot{y}_2 = 10 y_1 - y_2
				\end{cases}
			\end{equation}
		\end{example}
		
		\begin{example}
			\begin{equation}
				\begin{bmatrix}
					\dot{x}_1 \\
					\dot{x}_2
				\end{bmatrix} 
				= 
				\begin{bmatrix}
					-20  & 7 \\
					10 & -1
				\end{bmatrix}
				\begin{bmatrix}
					x_1 \\
					x_2 
				\end{bmatrix} 
			+
			\begin{bmatrix}
				1    \\
				0  
			\end{bmatrix}
			u
			\end{equation}
		\end{example}
		
	\end{flushleft}
\end{frame}



\begin{frame}{ODE and State-Space}
	%\framesubtitle{n-th order}
	\begin{flushleft}
		
		General form of an n-th order linear ODE with an input can be presented as follows:
		%
		\begin{equation}
			\label{eq:ODE_1}
			a_n y^{(n)} + 
			... +
			a_2 \ddot{y} + a_1 \dot{y} + 
			a_0 y = u(t)
		\end{equation}
	
	\bigskip
		
		The state-space representation of a linear system with an input is:
		%
		\begin{equation}
			\label{eq:SS_1}
			\dot{\bo{x}} = \bo{A} \bo{x} + \bo{B} \bo{u}
		\end{equation}
		
		Note that in \eqref{eq:ODE_1} $u$ is a scalar, while in \eqref{eq:SS_1} $\bo{u}$ can either be a scalar or a vector.
		
	\end{flushleft}
\end{frame}







\begin{frame}{Equations with an output}
	%\framesubtitle{n-th order}
	\begin{flushleft}
		
		Equations can also have an output. What "output" means depends on the particular use-case - it is not a mathematical issue, it is a question of interpretation. For example, an output can mean:
		
		\begin{itemize}
			\item What we measure (the position and orientation of a quadrotor, angular velocity of a motor's rotor, etc.).
			
			\item What we care about and/or what we want to control (the height of a quadrotor, the velocity of a car, etc.)
			
			\item etc.
		\end{itemize}
		
		We often denote output as $y$, and it depends on the state of the system: $y = g(\bo{x})$
		
	\end{flushleft}
\end{frame}


\begin{frame}{Equations with an output}
	%\framesubtitle{n-th order}
	\begin{flushleft}
		
		State-space representation of a linear system with an input and an output is:
		%
		\begin{equation}
		\begin{cases}
				\dot{\bo{x}} = \bo{A} \bo{x} + \bo{B}\bo{u} \\
				\bo{y} = \bo{C}\bo{x}
		\end{cases}
		\end{equation}
		
		If $\bo{u} \in \R$ and $\bo{y} \in \R$ (i.e. if they are scalars) and you want to represent the system with an output as a single ODE, it is typical to treat the output as the ODE variable:
		
		\begin{equation}
			a_n y^{(n)} + 
			... +
			a_2 \ddot{y} + a_1 \dot{y} + 
			a_0 y = u(t)
		\end{equation}
		
		
		If $\bo{u}$ and $\bo{y}$ are scalars, the system is called \emph{single-input single-output (SISO)}, if they are vectors - \emph{multi-input multi-output (MIMO)}.
		
		\bigskip
		
		We can always express a SISO system in either form - ODE or state-space.
		
	\end{flushleft}
\end{frame}





\begin{frame}{ODE to State-Space conversion}
% \framesubtitle{...are what we will study}
\begin{flushleft}

Consider eq. $\dddot{y} + a_2 \ddot{y} + a_1 \dot{y} + a_0 y =u$.

\bigskip

Make a substitution: $x_1 = y$, $x_2 = \dot{y}$, $x_3 = \ddot{y}$. We get:

\begin{align}
        \dot{x}_1 &= \dot{y} = x_2 \\
        \dot{x}_2 &= \ddot{y} = x_3 \\
        \dot{x}_3 &=  \dddot{y} = 
        u-a_2 \ddot{y} - a_1 \dot{y} - a_0 y = 
        u-a_2 x_3 - a_1 x_2 - a_0 x_1
\end{align}

Which can be directly put in the state-space form:

\begin{equation}
\begin{bmatrix}
\dot{x}_1 \\ \dot{x}_2 \\ \dot{x}_3
\end{bmatrix} 
=
\begin{bmatrix}
0 & 1 & 0 \\ 
0 & 0 & 1 \\
-a_0 & -a_1 & -a_2
\end{bmatrix} 
\begin{bmatrix}
x_1 \\ x_2 \\ x_3
\end{bmatrix} 
+ 
\begin{bmatrix}
0 \\ 0 \\ u
\end{bmatrix}
\end{equation}


\end{flushleft}
\end{frame}






\begin{frame}{Controllable canonical form}
	% \framesubtitle{...are what we will study}
	\begin{flushleft}
		
		In general, the following form (the result of a transformation of the of an ODE to a state-space form) is called controllable canonical form:
		
		\begin{align}
			\begin{cases}
			\dot{\bo{x}}
			&=
			\begin{bmatrix}
				0             & 1              & 0   & ... & 0 \\ 
				0             & 0             & 1    & ... & 0 \\
				...            & ...             & ...  & ... & ...  \\
				-a_{n-1} & -a_{n-2} & -a_{n-3} & ...  & -a_0
			\end{bmatrix} 
			\bo{x}
			+ 
			\begin{bmatrix}
				0 \\ 0\\ ... \\ 1
			\end{bmatrix}
			u
			\\
			y &= 			
			\begin{bmatrix}
				c_{n-1} & c_{n-1} &   ... & c_0
			\end{bmatrix}
			\bo{x}
		\end{cases}
		\end{align}
		
		
	\end{flushleft}
\end{frame}




\begin{frame}{Further reading}

\begin{itemize}
	
	
	\item K. Ogata Modern Control Engineering (Chapter 2.4)
	
\item Nise, N.S. Control systems engineering. John Wiley \& Sons. (Chapter 3 Modeling in Time Domain)	
	
\item Dorf \& Bishop, Modern Control Systems (Chapter 3.3 State differential equation)		
	
\item 2.14 Analysis and Design of Feedback Control Systems:

\begin{itemize}
	\item  \bref{http://web.mit.edu/2.14/www/Handouts/StateSpace.pdf}{State-Space Representation of LTI Systems}
	
	\item  \bref{http://web.mit.edu/2.14/www/Handouts/StateSpaceResponse.pdf}{Time-Domain Solution of LTI State Equations}
\end{itemize}	
	
\item \bref{https://lpsa.swarthmore.edu/}{Linear Physical Systems Analysis}:

\begin{itemize}
\item State Space Representations of Linear Physical Systems \bref{https://lpsa.swarthmore.edu/Representations/SysRepSS.html}{lpsa.swarthmore.edu/Representations/SysRepSS.html}

\item Transformation: Differential Equation to State Space \bref{https://lpsa.swarthmore.edu/Representations/SysRepTransformations/DE2SS.html}{lpsa.swarthmore.edu/.../DE2SS.html}
\end{itemize}	

\end{itemize}

\end{frame}



\myqrframe



\begin{frame}{State Space to ODE}
	%\framesubtitle{part 5}
	\begin{flushleft}
		
		\textcolor{blue}{\href{https://github.com/SergeiSa/Control-Theory-Slides-Spring-2022/blob/main/ColabNotebooks/StateSpace2ODE.ipynb}{Check out the code implementation.}}
		
		\bigskip
		
		
		\centerline{\textcolor{black}{\qrcode[height=2.1in]{https://github.com/SergeiSa/Control-Theory-Slides-Spring-2022/blob/main/ColabNotebooks/StateSpace2ODE.ipynb}}}
		
		
	\end{flushleft}
\end{frame}




\begin{frame}{Appendix}
	\framesubtitle{State-space to ODE conversion}
	\begin{flushleft}
		
		We can further generalize the ODE to the following form:
		%
		\begin{equation}
			a_n y^{(n)} + 
			... +
			a_2 \ddot{y} + a_1 \dot{y} + 
			a_0 y = h_{n-1} u^{(n-1)} + ... + h_0 u
		\end{equation}
		
		We can represent any\footnotemark state-space SISO in this form. 
		
		\bigskip
		
		First, we differentiate $y = \bo{C}\bo{x}$ $n$ times:
		%
		\begin{align}
			\dot{y} &= \bo{C}\dot{\bo{x}} = \bo{C}\bo{A} \bo{x} + \bo{C}\bo{B}u
			\\
			\ddot{y} &= \bo{C}\bo{A} \dot{\bo{x}} + \bo{C}\bo{B}\dot{u}  = 
			\bo{C}\bo{A}^2 \bo{x} + \bo{C}\bo{A}\bo{B}u + \bo{C}\bo{B}\dot{u} 
			\\
		    &	...
			\\
			y^{(n)} &= \bo{C}\bo{A}^n \bo{x} + \bo{C}\bo{A}^{n-1}\bo{B}u + ... + \bo{C}\bo{B}u^{(n-1)}
		\end{align}
		

		
		\footnotetext[1]{the system needs to be observable.} 
		
	\end{flushleft}
\end{frame}





\begin{frame}{Appendix}
	\framesubtitle{State-space to ODE conversion}
	\begin{flushleft}
		
		
		Next, we write down the characteristic equation of $\bo{A}$:
		%
		\begin{align}
			\text{det}(s\bo{I} - \bo{A}) = 0
			\\
			p(s) = s^n + \alpha_{n-1} s^{n-1} + ... +  \alpha_0 = 0
		\end{align}
		
		According to Cayley-Hamilton theorem, $p(\bo{A}) = \bo{0}$. Therefore:
		%
		\begin{align}
			p(\bo{A}) &= \bo{A}^n + \alpha_{n-1} \bo{A}^{n-1} + ... +  \bo{I}\alpha_0 =  \bo{0}
		\end{align}		
		
		With that, we can express $\bo{A}^n$ as a function of the lower powers:
		%
		\begin{align}
			\bo{A}^n = -\alpha_{n-1} \bo{A}^{n-1} - ... -  \bo{I}\alpha_0 
		\end{align}		
		
		
	\end{flushleft}
\end{frame}




\begin{frame}{Appendix}
	\framesubtitle{State-space to ODE conversion}
	\begin{flushleft}
		
		
		Earlier we found the term $ \bo{C}\bo{A}^n \bo{x}$ in the expression for $y^{(n)} $, the only term containing the state variable $\bo{x}$. We expand it using Cayley-Hamilton:
		%
		\begin{align}
			\bo{C}\bo{A}^n \bo{x} = -\alpha_{n-1}  \bo{C}\bo{A}^{n-1} \bo{x} - ... -  \alpha_0  \bo{C}\bo{x}
		\end{align}		
		
		But $\bo{C}\bo{x} = y$, ... , $\bo{C}\bo{A} \bo{x} = \dot{y} -  \bo{C}\bo{B}u$, etc. Substituting this to the expression for $y^{(n)}$ we find an ODE in terms fo input-output variables only (state variables were eliminated):
		%
		\begin{align}
			y^{(n)} &= m_{n-1} y^{(n-1)} + ... + m_0 y + p_{n-1} u^{(n-1)} + ... + p_0 u
		\end{align}		
		
	\end{flushleft}
\end{frame}


\end{document}
