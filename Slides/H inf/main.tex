\documentclass{beamer}

\input{settings.tex}


\title{H-infinity control}
\subtitle{Control Theory, Lecture ??}
\author{by Sergei Savin}
\centering
\date{\mydate}



\begin{document}
	\maketitle
	
	
	
%	\begin{frame}{Content}
%		\begin{itemize}
%			\item LMI
%			\item Control design
%			\item Quadratic Stability
%			\item Robustness
%			\item S-procedure
%			\item Appendix A
%		\end{itemize}
%	\end{frame}
	
	
	
	\begin{frame}{Singular values}
		\begin{flushleft}
			
			Consider the system:
			
			\begin{equation}
			\begin{cases}
				\dot{\bo{x}} = \bo{A}\bo{x}+\bo{B}\bo{w}
				\\
				\bo{y} = \bo{C}\bo{x}
			\end{cases}
			\end{equation}
			
			Its trasnfer function is $G(s) = \bo{C} (s\bo{I} - \bo{A})^{-1} \bo{B}$. The frequency responce at $\omega = \omega_0$ is:
			
			\begin{equation}
				G(j\omega_0) = \bo{C} (j\omega_0\bo{I} - \bo{A})^{-1} \bo{B}
			\end{equation}
			
			The spectral norm of the frequency responce matrix (its largest singular value) represents the largest input amplification that is possible at the frequency $ \omega_0$:
			
			\begin{equation}
				||G(j\omega_0)||_2 =\ \bar{\sigma} (G(j \omega_0))
			\end{equation}
			
			
		\end{flushleft}
	\end{frame}
	
	
		
	\begin{frame}{What is H-infinity}
		\begin{flushleft}
			
			
			The term "H-infinity" is a reference to the H-infinity norm of the system - the largest input amplification the system can produce at any frequency:
			
			\begin{equation}
				||G||_{\infty} = \underset{\omega}{\sup} \ \bar{\sigma} (G(j \omega))
			\end{equation}
			
			H-infinity is a control design strategy that provides a guarantee that even in the worst-case scenario, the input (disturbance) will not be amplified by more than a given amount $\gamma$.
			
		\end{flushleft}
	\end{frame}
	
	
	
	\begin{frame}{H-infinity conditions}
		\begin{flushleft}
			
			For the system $G(s)$:

			\begin{equation}
				\begin{cases}
					\dot{\bo{x}} = \bo{A}\bo{x}+\bo{B}\bo{w}
					\\
					\bo{y} = \bo{C}\bo{x}
				\end{cases}
			\end{equation}

		The following conditions are equibalent (by definition of H-inf and singular values):
		
		\begin{equation}
			||G||_\infty < \gamma \ \Longleftrightarrow \ G(j\omega)\T G(j\omega) \prec \gamma^2 I, \ \forall \omega
		\end{equation}
		
		KYP (Kalman–Yakubovich–Popov) lemma:
		
		\begin{equation}
			||G||_\infty < \gamma \ \Longleftrightarrow \ \int_0^\infty \left(\bo{y}\T \bo{y} - \gamma^2 \bo{w}\T \bo{w} \right) dt \leq 0
		\end{equation}
			
		\end{flushleft}
	\end{frame}
	
	
	
	
	\begin{frame}{Storage function criterion}
		\begin{flushleft}
			
			We can guarantee $||G||_\infty < \gamma$ if we find a \emph{storage function} $V( \bo{x}) > 0$ for all $\bo{x} \neq \bo{0}$ and $V(\bo{0}) = 0$, such that:
			
			\begin{equation}
				\dot V(\bo{x}) + \bo{y}\T \bo{y} - \gamma^2 \bo{w}\T \bo{w} \leq 0
			\end{equation}
			
			The quantity $\bo{y}\T \bo{y} - \gamma^2 \bo{w}\T \bo{w} $ is called \emph{supply rate}.
			
		\end{flushleft}
	\end{frame}
	
	
	\begin{frame}{Storage function criterion - proof}
		\begin{flushleft}
			
			Integrate the criterion from $t=0$ to $t = T$:
			
			\begin{equation}
				V(\bo{x}(T)) - V(\bo{x}(0)) + \int_{0}^{T} \bo{y}\T \bo{y} - \gamma^2 \bo{w}\T \bo{w} \leq 0
			\end{equation}
			
			Assume zero initial conditions (we consider pure input-output behaviour), so $V(\bo{x}(0)) = 0$ and remember that $V > 0$:
			
			\begin{equation}
				\int_{0}^{T} \bo{y}\T \bo{y} - \gamma^2 \bo{w}\T \bo{w}  \leq  -V(\bo{x}(T)) \leq 0
			\end{equation}
			
			Since this is true for all $T$, taking $T \rightarrow \infty$ recovers the earlier time-domain KYP criterion.
			
		\end{flushleft}
	\end{frame}
	
	
	
	
	\begin{frame}{Derivation of LMI conditions, 1}
		\begin{flushleft}
			
			Consider a storage function takes form $V( \bo{x}) =  \bo{x}\T  \bo{P}  \bo{x}$, where $ \bo{P} \succ 0$. Then the criterion takes the form:
			
			\begin{align}
				\dot{\bo{x}}\T  \bo{P}  \bo{x} + \bo{x}\T  \bo{P}  \dot{\bo{x}} + \bo{y}\T \bo{y} - \gamma^2 \bo{w}\T \bo{w} \leq 0
			\end{align}
			
			Substitute the dynamical equations:
			%
			\begin{align*}
				(\bo{A}\bo{x}+\bo{B}\bo{w})\T  \bo{P}  \bo{x} + \bo{x}\T  \bo{P} (\bo{A}\bo{x}+\bo{B}\bo{w}) + \bo{x}\T  \bo{C}\T \bo{C}\bo{x} - \gamma^2 \bo{w}\T \bo{w} \leq 0
			\end{align*}
			
			In matrix form:
			%
			\begin{align}
				\begin{bmatrix}
					\bo{x} \\ \bo{w}
				\end{bmatrix}\T
				\begin{bmatrix}
					\bo{A}\T\bo{P}+\bo{P}\bo{A}+\bo{C}\T\bo{C} & \bo{P}\bo{B} \\
					 \bo{B}\T\bo{P}  & -\gamma^2 \bo{I}
				\end{bmatrix}
				\begin{bmatrix}
					\bo{x} \\ \bo{w}
				\end{bmatrix}
				\leq 0
			\end{align}
			
			
			
		\end{flushleft}
	\end{frame}
	
	
	
	
	\begin{frame}{Derivation of LMI conditions, 2}
		\begin{flushleft}
			
			The last inequality needs to hold for all $\bo{x}$, $\bo{w}$, so it can be rewritten as an LMI:
			%
			\begin{align}
				\begin{bmatrix}
					\bo{A}\T\bo{P}+\bo{P}\bo{A}+\bo{C}\T\bo{C} & \bo{P}\bo{B} \\
					\bo{B}\T\bo{P}  & -\gamma^2 \bo{I}
				\end{bmatrix}
				\prec 0
			\end{align}
			
			This is the standard LMI for checking the H-infinity norm of a system.
			
		\end{flushleft}
	\end{frame}
	
	
	\begin{frame}{H-infinity control design, 1}
		\begin{flushleft}
			
			Consider the system:
			
			\begin{equation}
				\begin{cases}
					\dot{\bo{x}} = \bo{A}\bo{x}+\bo{B}_1\bo{u}+\bo{B}_2\bo{w}
					\\
					\bo{y} = \bo{C}\bo{x}
				\end{cases}
			\end{equation}
			
			Assume linear control $\bo{u} = \bo{K}\bo{x}$, the first equation then becomes $\dot{\bo{x}} = (\bo{A}+\bo{B}_1\bo{K})\bo{x}+\bo{B}_2\bo{w}$. We can plug it to the earlier derived conditions:
			%
			\begin{align}
				\begin{bmatrix}
					(\bo{A}+\bo{B}_1\bo{K})\T\bo{P}+\bo{P}(\bo{A}+\bo{B}_1\bo{K})+\bo{C}\T\bo{C} & \bo{P}\bo{B}_2 \\
					\bo{B}_2\T\bo{P}  & -\gamma^2 \bo{I}
				\end{bmatrix}
				\prec 0
			\end{align}
			
			Now we need to turn this into an LMI (linear in decision variables).
			
		\end{flushleft}
	\end{frame}



\begin{frame}{H-infinity control design, 2}
	\begin{flushleft}
		
		We divide the inequality by $\gamma$ and define $\bar{\bo{P}} = \frac{1}{\gamma} \bo{P}$:
		
		\begin{align}
			\begin{bmatrix}
				(\bo{A}+\bo{B}_1\bo{K})\T\bar{\bo{P}}+\bar{\bo{P}}(\bo{A}+\bo{B}_1\bo{K})+\frac{1}{\gamma}\bo{C}\T\bo{C} & \bar{\bo{P}}\bo{B}_2 \\
				\bo{B}_2\T\bar{\bo{P}}  & -\gamma \bo{I}
			\end{bmatrix}
			\prec 0
		\end{align}
		
		Conjugate by $\begin{bmatrix}
			\bo{Q} &\bo{0} \\
			\bo{0}  &  \bo{I}
		\end{bmatrix}$, where $\bo{Q} = \bar{\bo{P}}^{-1}$:
		
		\begin{align}
			\begin{bmatrix}
				\bo{Q}(\bo{A}+\bo{B}_1\bo{K})\T+(\bo{A}+\bo{B}_1\bo{K})\bo{Q}+\frac{1}{\gamma}\bo{Q}\bo{C}\T\bo{C}\bo{Q} & \bo{B}_2 \\
				\bo{B}_2\T  & -\gamma \bo{I}
			\end{bmatrix}
			\prec 0
		\end{align}
		
		Apply Schur and define $\bo{L} = \bo{K}\bo{Q}$ to get the LMI:
		
		\begin{align}
			\begin{bmatrix}
				\bo{Q}\bo{A}\T + \bo{A}\bo{Q} + \bo{L}\T \bo{B}_1\T +\bo{B}_1 \bo{L} & \bo{B}_2 & \bo{Q}\bo{C}\T \\
				\bo{B}_2\T  & -\gamma \bo{I} & \bo{0} \\
				\bo{C}\bo{Q} & \bo{0} & -\gamma \bo{I}
			\end{bmatrix}
			\prec 0
		\end{align}
		
		
	\end{flushleft}
\end{frame}
	
	
	
	
	
\myqrframe
	
	
	
	
	\begin{frame}{Appendix A}
		\framesubtitle{Congruence transformation and definiteness}
		\begin{flushleft}
			
			Consider matrices $\bo{P} \succ 0$, and $\bo{V} \in \R^{n, n}$, the latter is full rank. We can prove that:
			
			\begin{equation}
				\bo{P} \succ 0 \implies \bo{V}^\top\bo{P}\bo{V} \succ 0
			\end{equation}
			
			Proof: $\bo{x}^\top\bo{V}^\top\bo{P}\bo{V}\bo{x} = \bo{z}^\top\bo{P}\bo{z}$, where $\bo{z} = \bo{V}\bo{x}$. Since $\bo{P} \succ 0$, $\bo{z}^\top\bo{P}\bo{z} \geq 0$, hence $\bo{x}^\top\bo{V}^\top\bo{P}\bo{V}\bo{x} \geq 0$. 
			
			\begin{definition}
				Congruence transformation preserves semi-definiteness: $\text{det}(\bo{V}) \neq 0, \ \bo{P} \succ 0 \implies \bo{V}^\top\bo{P}\bo{V} \succ 0$
			\end{definition}
			
			
		\end{flushleft}
	\end{frame}
	
	
	
	
\end{document}
