\documentclass{beamer}

\input{settings.tex}


\title{Discrete Dynamics}
\subtitle{Control Theory, Lecture 6}
\author{by Sergei Savin}
\centering
\date{\mydate}



\begin{document}
\maketitle


\begin{frame}{Content}

\begin{itemize}
\item Discrete Dynamics
\item Stability of the Discrete Dynamics
\item CT-LTI: Analytical solution
\begin{itemize}
    \item Matrix exponential
    \item Analytical solution
    \item Forced response
\end{itemize}
\item Discretization
\begin{itemize}
	\item Exact
	\item Zero-order hold
\end{itemize}
\item Read more
\end{itemize}

\end{frame}





\begin{frame}{Discrete Dynamics}
% \framesubtitle{Part 1}
\begin{flushleft}

The following dynamical system is called \emph{discrete}:

\begin{equation}
    \bo{x}_{i+1} = \bo{A}\bo{x}_i + \bo{B}\bo{u}_i
\end{equation}


\begin{itemize}
    \item There are no derivatives in the equation;
    \item It is easy to simulate.
\end{itemize}

\bigskip

The affine control for this system can be written as:

\begin{equation}
    \bo{u}_i = -\bo{K}\bo{x}_i + \bo{u}_i^*
\end{equation}

\end{flushleft}
\end{frame}




\begin{frame}{Stability of the Discrete Dynamics}
%	\framesubtitle{Real eigenvalues}
	\begin{flushleft}
		
		Let us consider stability of the discrete dynamical system where matrix $\bo{A}$ has purely real eigenvalues:
		
		\begin{equation}
			\bo{x}_{i+1} = \bo{A}\bo{x}_i
		\end{equation}		
		
		With eigendecomposition $\bo{A} = \bo{V}^{-1} \bo{D} \bo{V}$ (where $\bo{D}$ is a diagonal matrix with eigenvalues $\lambda_j$ of $\bo{A}$ on its diagonal)  and introducing notation $\bo{z}_i = \bo{V}\bo{x}_i$ we get:
		
		\begin{equation}
			\bo{x}_{i+1} = \bo{V}^{-1} \bo{D} \bo{V}\bo{x}_i
		\end{equation}
		\begin{equation}
			\bo{z}_{i+1} =  \bo{D} \bo{z}_i
		\end{equation}
		
		Meaning that the dynamics became a system of independent scalar equations $z_{j, i+1} =  \lambda_j z_{j, i}$. 
		
		
	\end{flushleft}
\end{frame}




\begin{frame}{Stability of the Discrete Dynamics}
	\framesubtitle{Real eigenvalues}
	\begin{flushleft}
		
		Considering individual equations $z_{j, i+1} =  \lambda_j z_{j, i}$ we find that the absolute value of the scalars $z_{j}$ will dwindle with time iff $|\lambda_j| < 1$:
		
		\begin{equation}
			\left| \frac{z_{j, i+1}}{z_{j, i}} \right | =   | \lambda_j |
		\end{equation}
		 
		
	\end{flushleft}
\end{frame}



\begin{frame}{Stability of the Discrete Dynamics}
	\framesubtitle{2x2 system}
	\begin{flushleft}
		
		Let us consider stability of the discrete dynamical system with a 2-by-2 matrix $\bo{A}$:
		
		\begin{equation}
			\begin{bmatrix}
				x_{1, i+1} \\ x_{2, i+1}
			\end{bmatrix}
			= 
			\begin{bmatrix}
				\alpha & -\beta \\ \beta & \alpha
			\end{bmatrix}     
			\begin{bmatrix}
				x_{1, i} \\ x_{2, i}
			\end{bmatrix}
		\end{equation}	
		
		Let us find norms of $\begin{bmatrix}
			x_{1, i+1} \\ x_{2, i+1}
		\end{bmatrix}$ and $\begin{bmatrix}
		x_{1, i} \\ x_{2, i}
	\end{bmatrix}$:


		\begin{equation}
		\left | \left|
		\begin{bmatrix}
			x_{1, i} \\ x_{2, i}
		\end{bmatrix}
		\right | \right|^2
			= 
	 	x_{1, i}^2 + x_{2, i}^2
		\end{equation}

		\begin{equation}
			\left | \left|
			\begin{bmatrix}
				x_{1, i+1} \\ x_{2, i+1}
			\end{bmatrix}
			\right | \right|^2
			= 
			(\alpha^2 + \beta^2) (x_{1, i}^2 + x_{2, i}^2)
		\end{equation}
		
		
	\end{flushleft}
\end{frame}



\begin{frame}{Stability of the Discrete Dynamics}
	\framesubtitle{2x2 system}
	\begin{flushleft}
		
		We can find the ratio of the norms of $\begin{bmatrix}
			x_{1, i+1} \\ x_{2, i+1}
		\end{bmatrix}$ and $\begin{bmatrix}
			x_{1, i} \\ x_{2, i}
		\end{bmatrix}$: 
		
		
		
		\begin{equation}
			\left | \left|
			\begin{bmatrix}
				x_{1, i+1} \\ x_{2, i+1}
			\end{bmatrix}
			\right | \right|^2 
			/ 
			\left | \left|
			\begin{bmatrix}
				x_{1, i} \\ x_{2, i}
			\end{bmatrix}
			\right | \right|^2
			= 
			\alpha^2 + \beta^2
		\end{equation}
		
		Remembering that eigenvalues of the system are $\lambda  = \alpha \pm j\beta$, we can rewrite the expression above as:
		
		\begin{equation}
	\left | \left|
	\begin{bmatrix}
		x_{1, i+1} \\ x_{2, i+1}
	\end{bmatrix}
	\right | \right|^2 
	/ 
	\left | \left|
	\begin{bmatrix}
		x_{1, i} \\ x_{2, i}
	\end{bmatrix}
	\right | \right|^2
	= 
	| \lambda |^2
		\end{equation}		
		
		We can see that the norm of the variable $\bo{x}$ will dwindle iff $|\lambda| < 1$.
		
	\end{flushleft}
\end{frame}




\begin{frame}{Stability of the Discrete Dynamics}
\begin{flushleft}

General stability criterion is given below:

\bigskip

\begin{block}{Stability criterion}
Discrete system $\bo{x}_{i+1} = \bo{A}\bo{x}_i$ is stable as long as the eigenvalues of $\bo{A}$ are smaller than 1 by absolute value: $|\lambda_i(\bo{A})| \leq 1, \; \forall i$.
\end{block}

\end{flushleft}
\end{frame}



\begin{frame}
	
	\centering{\huge CT-LTI: Analytical solution}
	
\end{frame}





\begin{frame}{Matrix exponential}
	% \framesubtitle{Limited control}
	\begin{flushleft}
		
		Exponential $e^a$ is defined as a series:
		
		\begin{equation}
			e^a =1+a+\frac{1}{2} a^2
			+\frac{1}{6} a^3 + ... = \sum_{n=0}^{\infty} \frac{1}{n!} a^n
		\end{equation}
		
		Matrix exponential $e^\bo{A}$ is defined as a series:
		
		\begin{equation}
			e^\bo{A} = \bo{I}+\bo{A}+\frac{1}{2} \bo{A}\bo{A}
			+\frac{1}{6} \bo{A}\bo{A}\bo{A} + ... = \sum_{n=0}^{\infty} \frac{1}{n!} \bo{A}^n
		\end{equation}
		
		
	\end{flushleft}
\end{frame}




\begin{frame}{Analytical solution to ODE}
	% \framesubtitle{Limited control}
	\begin{flushleft}
		
		An ODE of the form $\dot x = a x$ has analytical solution $x(t) = e^{at} x(0)$.
		
		\bigskip
		
		An ODE of the form $\dot{\bo{x}} = \bo{A}  \bo{x}$ has analytical solution $\bo{x}(t) = e^{\bo{A}t}  \bo{x}(0)$.
		
		\bigskip
		
		Let us check that this is a solution:
		
		\begin{align}
			\bo{x}(t) &= \left( \bo{I}+\bo{A}t+\frac{1}{2} \bo{A}\bo{A}t^2
			+\frac{1}{6} \bo{A}\bo{A}\bo{A}t^3 + ... \right)  \bo{x}(0) &
			\\
			\dot{\bo{x}}(t) &= \left( \bo{A}+\bo{A}\bo{A}t
			+\frac{1}{2}\bo{A}\bo{A}\bo{A}t^2 + ... \right)  \bo{x}(0) 	&		
			\\
			\dot{\bo{x}}(t) &= \bo{A} \left(\bo{I} +\bo{A}t
			+\frac{1}{2}\bo{A}\bo{A}t^2 + ... \right)  \bo{x}(0) 	&		
			\\
			\dot{\bo{x}}(t) &= \bo{A} e^{\bo{A}t}  \bo{x}(0) 	&		 			
			\\
			\dot{\bo{x}}(t) &= \bo{A} \bo{x}(t)  &\ \ \qed
		\end{align}
		
		
		
	\end{flushleft}
\end{frame}



\begin{frame}{Forced State Response, 1}
	% \framesubtitle{Limited control}
	\begin{flushleft}
		
		An ODE of the form $\dot x = a x + bu(t)$ also has analytical solution. To find it, we first find the following derivative:
		%
		\begin{equation}
			\frac{d}{dt} \left( e^{-at} x(t) \right) = e^{-at} \dot x(t) - a e^{-at} x(t)
		\end{equation}
		
		Multiplying $\dot x = a x + bu(t)$ by $e^{-at}$ we see:
		
		\begin{align}
			e^{-at} \dot x = e^{-at} a x + e^{-at} bu(t)
			\\
			e^{-at} \dot x -  e^{-at} a x= e^{-at} bu(t)
			\\
			\frac{d}{dt} \left( e^{-at} x(t) \right) = e^{-at} bu(t)
			\\
			\int_{0}^{t} \frac{d}{d\tau} \left( e^{-a\tau} x(\tau) \right) d\tau = 	\int_{0}^{t}  e^{-a\tau} bu(\tau) d\tau
		\end{align}
		
		
		
	\end{flushleft}
\end{frame}



\begin{frame}{Forced State Response, 2}
	% \framesubtitle{Limited control}
	\begin{flushleft}
		
		Continuing the derivation:
		
		\begin{align}
			\int_{0}^{t} \frac{d}{d\tau} \left( e^{-a\tau} x(\tau) \right) d\tau = 	\int_{0}^{t}  e^{-a\tau} bu(\tau) d\tau
			\\
			e^{-at} x(t) - x(0) = \int_{0}^{t}  e^{-a\tau} bu(\tau) d\tau
			\\
			x(t) = e^{at} x(0) + e^{at} \int_{0}^{t}  e^{-a\tau} bu(\tau) d\tau
			\\
			x(t) = e^{at} x(0) + \int_{0}^{t}  e^{a(t-\tau)} bu(\tau) d\tau
		\end{align}
		
		
		
	\end{flushleft}
\end{frame}




\begin{frame}{Forced State Response, 3}
	% \framesubtitle{Limited control}
	\begin{flushleft}
		
		State-space equation $\dot{\bo{x}} = \bo{A}  \bo{x} + \bo{B}  \bo{u}(t)$ also has an analytical solution:
		
		\begin{equation}
			\bo{x}(t) = e^{\bo{A}t}  \bo{x}(0) + 
			\int_{0}^{t} e^{\bo{A}(t-\tau)} \bo{B}  \bo{u}(\tau) d\tau
		\end{equation}
		
		The same can be re-written as:
		
		\begin{equation}
			\bo{x}(t) = e^{\bo{A}t}  \bo{x}(0) + 
			e^{\bo{A}t} \int_{0}^{t} e^{-\bo{A}\tau} \bo{B}  \bo{u}(\tau) d\tau
		\end{equation}
		
	\end{flushleft}
\end{frame}






\begin{frame}
	
		\centering{\huge Discretization}
	
\end{frame}





\begin{frame}{From analytical solution to discrete dynamics}
	% \framesubtitle{Limited control}
	\begin{flushleft}
		Given a solution to a state-space system we can consider how the system evolves from the point $t=0$ to the point $t=\Delta t$, assuming that $\bo{u}(t) = \bo{u}_0 = \text{const}, \ t \in [ 0, \ \Delta t ]$, and denoting $\bo{x}(0) = \bo{x}_0$ and $\bo{x}(\Delta t) = \bo{x}_1$:
		
		\begin{equation}
			\bo{x}_1 = e^{\bo{A}\Delta t}  \bo{x}_0 + 
			e^{\bo{A} \Delta t} \int_{0}^{\Delta t} e^{-\bo{A}\tau} \bo{B}  \bo{u}_0 d\tau
		\end{equation}
		
		Now we denote:
		%
		\begin{align}
			\bar{\bo{A}} = e^{\bo{A}\Delta t}, \ \ \
			\bar{\bo{B}} = e^{\bo{A} \Delta t} \int_{0}^{\Delta t} e^{-\bo{A}\tau} \bo{B} d\tau
		\end{align}
	
		We get:
		%
		\begin{equation}
			\bo{x}_1 = \bar{\bo{A}}  \bo{x}_0 + \bar{\bo{B}}  \bo{u}_0
		\end{equation}		
		
	\end{flushleft}
\end{frame}






\begin{frame}{CT and DT eigenvalues}
	\begin{flushleft}
		
		Consider a system $\dot{x} = \lambda x$ with the solution $x(t) = e^{ \lambda t} x(0)$. Discretization will look like:
		
		\begin{equation}
			x_1 =e^{ \lambda \Delta t}   x_0 
		\end{equation}
		
		Hence, DT eigenvalue is $\lambda_d = e^{ \lambda \Delta t} $. Assuming $\lambda < 0$, We get:
		
		\begin{itemize}
			\item As $\Delta t \rightarrow 0$, the DT eigenvalue $|\lambda_d| \rightarrow 1$;
			
			\item For $\Delta t >> 0$, the DT eigenvalue $|\lambda_d| \rightarrow 0$;
		\end{itemize}
		
	\end{flushleft}
\end{frame}







\begin{frame}{Approximate Discretization, 1}
%\framesubtitle{Finite difference in an autonomous LTI}
\begin{flushleft}

We can expand the exact solution $\bar{\bo{A}} = e^{\bo{A}\Delta t}$ through the definition of matrix exponential:
%
\begin{equation}
	\bar{\bo{A}} = e^{\bo{A}\Delta t} =  \sum_{n=0}^{\infty} \frac{1}{n!} \bo{A}^n \Delta t^n
\end{equation}

Dropping the highter-order terms we get:
%
\begin{equation}
\bo{x}_{i + 1} = ( \bo{I} + \bo{A} \Delta t) \bo{x}_i
\end{equation}

For smaller $\Delta t$ the higher-order terms (quadratic and higher in $\Delta t$) are approaching 0, making the approximation more accurate. 

\bigskip

\textcolor{mynormalgray}
{
	The same could be derived by replacing the derivative in $\dot{\bo{x}} = \bo{A}  \bo{x}$ with a dinite-difference aproximation:}
%
\begin{equation*}
	\textcolor{mynormalgray}
	{\dot {\bo{x}} \approx \frac{1}{\Delta t}(\bo{x}_{i+1} - \bo{x}_i)}
\end{equation*}

\end{flushleft}
\end{frame}




\begin{frame}{Approximate Discretization, 2}
%\framesubtitle{Zero order hold}
\begin{flushleft}

Defining \emph{discrete state space matrix} $\bar{\bo{A}}$ and \emph{discrete control matrix} $\bar{\bo{B}}$ as follows:

\begin{equation}
\bar{\bo{A}} = \bo{A} \Delta t + \bo{I}
\end{equation}
\begin{equation}
\bar{\bo{B}} = \bo{B} \Delta t
\end{equation}
%
We get discrete dynamics:

\begin{equation}
\bo{x}_{i+1} = \bar{\bo{A}} \bo{x}_i + \bar{\bo{B}} \mathbf u_i
\end{equation}

\end{flushleft}
\end{frame}










\begin{frame}{Read more}

\begin{itemize}
	
	\item K. Ogata Modern Control Engineering (Chapter 9.4)
	
	\item Dorf \& Bishop, Modern Control Systems (Chapter 3.3 State differential equation)		
	
\item  \bref{http://cse.lab.imtlucca.it/~bemporad/teaching/ac/pdf/04a-TD_sys.pdf}{Automatic Control 1 Discrete-time linear systems}, Prof. Alberto Bemporad, University of Trento

\item \bref{https://web.mit.edu/2.14/www/Handouts/StateSpaceResponse.pdf}{MIT 2.14, State Space Response}

\end{itemize}

\end{frame}



\myqrframe




\begin{frame}{Jordan blocks and stability}
	% \framesubtitle{Limited control}
	\begin{flushleft}
		
		Consider a matrix $\bo{A} \in \R^{n \times n}$ with a single Jordan block of size $n$ (the entire matrix is a Jordan block). Any Jordan blcok can be writen as:
		%
		\begin{equation}
			\bo{A} = \lambda \bo{I} +\bo{N} 
		\end{equation}
		%
		where $\lambda$ is the eigenvalue and $\bo{N} $ is a nilpotent matrix, $\bo{N}^n = \bo{0}$.
		
		\bigskip
		
		We can solve the equation $\dot{\bo{x}} = \bo{A}\bo{x}$ using matrix exponentials:
		%
		\begin{align}
			\bo{x} (t)= e^{(\lambda \bo{I} +\bo{N})t  }= e^{\lambda t} e^{\bo{N}t} \bo{x} (0)=
			\\
			=  e^{\lambda t} \sum_{k=0}^{n-1} \frac{1}{k!} \bo{N}^k t^k \bo{x} (0)
		\end{align}
		
		If $\lambda = 0$, then the solution is polynomial:
		%
		\begin{align}
			\bo{x} (t)=  \sum_{k=0}^{n-1} \frac{1}{k!} \bo{N}^k t^k \bo{x} (0)
		\end{align}
		 
		
		
	\end{flushleft}
\end{frame}



\end{document}
