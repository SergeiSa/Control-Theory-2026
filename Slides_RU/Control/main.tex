\documentclass{beamer}

\usepackage[T2A]{fontenc}
\usepackage[utf8]{inputenc}
\usepackage[english,russian]{babel}
\input{settings.tex}


\title{Управление}
\subtitle{Теория управления, лекция 3}
\author{Сергей Савин}
\centering
\date{\mydate}



\begin{document}
\maketitle

\begin{frame}{Изменение устойчивости}
	% \framesubtitle{O}
	\begin{flushleft}
		
		Рассмотрим две LTI (Линейные Стационарные Системы):
		
		\begin{equation}
			\dot{x} = 2 x
		\end{equation}
		
		\begin{equation}
			\dot{x} = 2 x + u
		\end{equation}
		
		Первая система автономна и неустойчива. Вторая — неавтономна, и мы не можем узнать, сходится ли её решение к нулю, пока не узнаем, что такое $u$.
		
		\bigskip
		
		Если мы выберем $u=0$, то получится неустойчивое уравнение. Но мы также можем выбрать $u$ таким образом, чтобы результирующая динамика стала устойчивой, например $u=-3x$:
		
		\begin{equation}
			\dot{x} = 2 x + u = 2 x - 3x = -x
		\end{equation}
		
		\begin{block}{ }
			Таким образом, мы можем использовать \emph{управляющее воздействие} $u$, чтобы изменить устойчивость системы!
		\end{block}
		
	\end{flushleft}
\end{frame}

\begin{frame}{Стабилизирующее управление}
	% \framesubtitle{O}
	\begin{flushleft}
		
		\begin{definition}
			Задача нахождения закона управления $\bo{u}$, который делает определённое решение $\bo{x}^*$ динамической системы $\dot{\bo{x}} = \bo{f}(\bo{x}, \bo{u})$ устойчивым, называется \emph{задачей стабилизирующего управления}.
		\end{definition}
		
		\bigskip
		
		Это верно как для линейных, так и для нелинейных систем. Но для линейных систем мы можем получить гораздо больше деталей о решении этой задачи, если ограничим выбор закона управления.
		
	\end{flushleft}
\end{frame}

\begin{frame}{Линейное управление}
	\framesubtitle{Замкнутая система}
	\begin{flushleft}
		
		Рассмотрим LTI:
		
		\begin{equation}
			\dot{\bo{x}} = \bo{A}\bo{x} + \bo{B}\bo{u}
		\end{equation}
		
		и выберем \emph{управление как линейную функцию состояния} $x$:
		
		\begin{equation}
			\bo{u} = -\bo{K}\bo{x}
		\end{equation}
		
		Матрицу $\bo{K}$ мы называем \emph{коэффициентом усиления (gain)}. Таким образом, мы знаем, как будет выглядеть система при приложенном управлении:
		
		\begin{equation}
			\dot{\bo{x}} = \bo{A}\bo{x} - \bo{B}\bo{K}\bo{x}
		\end{equation}
		\begin{equation}
			\label{eq:closed_loop}
			\dot{\bo{x}} = (\bo{A} - \bo{B}\bo{K})\bo{x}
		\end{equation}
		
		Заметим, что \eqref{eq:closed_loop} — это автономная система. Мы называем её \emph{замкнутой системой}.
		
	\end{flushleft}
\end{frame}

\begin{frame}{Линейное управление}
	\begin{flushleft}
		
		Мы можем проанализировать устойчивость системы $\dot{\bo{x}} = (\bo{A} - \bo{B}\bo{K})\bo{x}$:
		
		\begin{block}{Условие устойчивости для замкнутой LTI}
			Действительные части собственных значений матрицы $(\bo{A} - \bo{B}\bo{K})$ должны быть отрицательными для асимптотической устойчивости, или неположительными для устойчивости по Ляпунову.
		\end{block}
		
		\begin{block}{Матрица Гурвица}
			Если квадратная матрица $\bo{M}$ имеет собственные значения со строго отрицательными действительными частями, она называется матрицей Гурвица. Мы будем обозначать это как $\bo{M} \in \mathcal{H}$.
		\end{block}
		
		%\bigskip
		
		Таким образом, всё, что нужно сделать, — это найти такую матрицу $\bo{K}$, чтобы $(\bo{A} - \bo{B}\bo{K})$ была матрицей Гурвица, и вы получили асимптотически устойчивую замкнутую систему!
		
	\end{flushleft}
\end{frame}

\begin{frame}{Скалярный случай}
	\begin{flushleft}
		
			Рассмотрим следующую систему:
		
		\begin{equation}
			\dot x = a x + b u
		\end{equation}
		
		мы можем выбрать следующий линейный закон управления: $u = - k x$. Замкнутая система для этого примера имеет вид:
		
		\begin{equation}
			\dot x = (a- bk) x
		\end{equation}		
		
		Решение для замкнутой системы:
		
		\begin{equation}
			x(t) =  x_0 e^{(a- bk)t}
		\end{equation}		
		
		Пока выполняется $a- bk < 0$, решение сходится к нулю. Поскольку мы можем выбирать $k$, мы можем подобрать его так, чтобы $a- bk = -q$, где $q$ — положительное число. Тогда мы выбираем $k = \frac{q+a}{b}$, получая устойчивую систему с собственным значением $-q$.
		
		\end{flushleft}
	\end{frame}



\begin{frame}{Многомерный случай}
	\begin{flushleft}
		
		Рассмотрим следующую систему:
		%
		\begin{equation}
			\begin{bmatrix}
				\dot x_1 \\ \dot x_2
			\end{bmatrix} 
			= 
			\begin{bmatrix}
				a_{11} & a_{12} \\ 0 & a_{22}
			\end{bmatrix}
			\begin{bmatrix}
				x_1 \\ x_2
			\end{bmatrix} 
			+ 
			\begin{bmatrix}
				b \\ 0
			\end{bmatrix}
			u
		\end{equation}
		
		С законом управления:
		%
		\begin{equation}
			u
			= 
			-
			\begin{bmatrix}
				k_1 & k_2
			\end{bmatrix}
			\begin{bmatrix}
				x_1 \\ x_2
			\end{bmatrix} 
		\end{equation}
		
		Замкнутая система имеет вид:
		%
		\begin{equation}
			\begin{bmatrix}
				\dot x_1 \\ \dot x_2
			\end{bmatrix} 
			= 
			\begin{bmatrix}
				a_{11}-b k_1 & a_{12}-b k_2 \\ 0 & a_{22}
			\end{bmatrix}
			\begin{bmatrix}
				x_1 \\ x_2
			\end{bmatrix} 
		\end{equation}
		
		Собственные значения замкнутой системы — это $a_{11}-b k_1$ и $a_{22}$. Второе собственное значение не может быть изменено выбором коэффициентов управления. Если $a_{22} < 0$, нам нужно выбрать $k_1$ так, чтобы $a_{11}-b k_1 = -q$, где $q$ — положительное число: $k_1 = \frac{q + a_{11}}{b}$.
		
	\end{flushleft}
\end{frame}



\begin{frame}{Пружина-масса-демпфер, 1}
	%\framesubtitle{Stability of the closed-loop system}
	\begin{flushleft}
		
		Рассмотрим систему пружина-масса-демпфер:
		%
		\begin{equation}
			\ddot y + \mu \dot y + c y = 0
		\end{equation}
		
		Уравнение может быть переписано в форме пространства состояний с заменой переменных $x_1 = y$ и $x_2 = \dot y$:
		%
		\begin{equation}
			\begin{bmatrix}
				\dot x_1 \\ \dot x_2
			\end{bmatrix} 
			= 
			\begin{bmatrix}
				0 & 1 \\ - c & -\mu
			\end{bmatrix}
			\begin{bmatrix}
				x_1 \\ x_2
			\end{bmatrix}
		\end{equation}
		
		
		Собственные значения матрицы 2x2 легко вычислить, используя её определитель $\text{det}$ и след $\text{tr}$:
		%
		\begin{equation}
			\lambda = \frac{\text{tr} \pm \sqrt{\text{tr}^2 - 4 \text{det}}}{2} 
		\end{equation}
		
		Здесь $\text{det} = c$ и $\text{tr} = -\mu$:
		%
		\begin{equation}
			\lambda = \frac{-\mu \pm \sqrt{\mu^2 - 4 c}}{2} 
		\end{equation}
		
	\end{flushleft}
\end{frame}


\begin{frame}{Пружина-масса-демпфер, 2}
	% \framesubtitle{O}
	\begin{flushleft}
		
		Проанализируем собственные значения $\lambda = \frac{-\mu \pm \sqrt{\mu^2 - 4 c} }{2}$. Мы видим, что если \textcolor{mydarkgreen}{$\mu > 0$} и \textcolor{mydarkgreen}{$ c > 0$}, возможны только два сценария: 
		
		\begin{enumerate}
			\item $\mu^2 - 4c \geq 0$, в этом случае $\sqrt{\mu^2 - 4c} \leq \mu$, собственные значения вещественные и \textcolor{mydarkgreen}{отрицательные}.
			\item $\mu^2 - 4c < 0$, в этом случае $\sqrt{\mu^2 - 4c}$ — чисто мнимое число, собственные значения комплексные с \textcolor{mydarkgreen}{отрицательными вещественными частями}.
		\end{enumerate}
		
		Если $\mu > 0$ и \textcolor{myblue}{$c = 0$}, то $\lambda_1 = -\mu$, $\lambda_2 = 0$, следовательно, система \textcolor{myblue}{нейтрально (маргинально) устойчива}.
		
		Если \textcolor{myblue}{$\mu = 0$} и $c > 0$, то $\lambda = \pm i \sqrt{c}$, следовательно, система \textcolor{myblue}{нейтрально (маргинально) устойчива}.
		
	\end{flushleft}
\end{frame}



\begin{frame}{Пружина-масса-демпфер, 3}
	\begin{flushleft}
		
		
		Если $\mu \geq 0$ и \textcolor{red}{$c < 0$}, то $\sqrt{\mu^2 - 4c} \geq \mu$, и собственные значения вещественные, причём одно из них \textcolor{red}{положительное}, система неустойчива. Если \textcolor{red}{$\mu < 0$} и \textcolor{red}{$c < 0$}, по крайней мере одно из собственных значений всё равно \textcolor{red}{положительное}.
		
		\bigskip
		
		Если \textcolor{red}{$\mu < 0$} и $c \geq 0$, то снова возможны только два сценария: 
		
		\begin{enumerate}
			\item $\mu^2 - 4c \geq 0$, в этом случае $\sqrt{\mu^2 - 4c} \leq \mu$, собственные значения вещественные и \textcolor{red}{положительные}.
			\item $\mu^2 - 4c < 0$, в этом случае $\sqrt{\mu^2 - 4c}$ — чисто мнимое число, собственные значения комплексные с \textcolor{red}{положительными вещественными частями}.
		\end{enumerate}
		
		\begin{definition}
			Система $\ddot y + \mu \dot y + c y = 0$ устойчива тогда и только тогда, когда $\mu \geq 0$ и $c \geq 0$.
		\end{definition}
		
	\end{flushleft}
\end{frame}



\begin{frame}{ПД-регулятор, 1}
	\begin{flushleft}
		
		Рассмотрим систему пружина-масса-демпфер:
		%
		\begin{equation}
			\ddot y + \mu \dot y + c y = u
		\end{equation}
		
		Мы можем предложить обратную связь в виде:
		%
		\begin{equation}
			u = -k_d \dot y  -k_p y 
		\end{equation}
		%
		Это называется \emph{пропорционально-дифференциальным регулятором}, часто сокращаемым как \emph{ПД-регулятор}; $k_d$ — дифференциальный коэффициент, а $k_p$ — пропорциональный коэффициент. Замкнутая система имеет вид:
		%
		\begin{align}
			\ddot y + (\mu + k_d) \dot y + (c+k_p) y = 0
		\end{align}
		
		Собственные значения:
		%
		\begin{align}
			\lambda &= \frac{-(\mu + k_d) \pm d}{2} 
			\\
			d &= \sqrt{(\mu + k_d)^2 - 4 (c+k_p)}
		\end{align}
		
		
	\end{flushleft}
\end{frame}



\begin{frame}{ПД-регулятор, 2}
	\begin{flushleft}
		
		Дано $c = 40$ и $\mu = 8$. Предположим, что мы хотим, чтобы замкнутая система имела собственные значения $\lambda_1 = -4$ и $\lambda_2 = -20$.
		%
		\begin{align}
			16 = \lambda_1 - \lambda_2= \frac{-(\mu + k_d) +d}{2} - \frac{-(\mu + k_d) - d}{2}  = d
		\end{align}
		%
		Отсюда следует:
		%
		\begin{align}
			-4 = \lambda_1 = \frac{-(\mu + k_d) + 16}{2}
			\\
			-(\mu + k_d) + 16 = -8
			\\
			k_d  = 16
		\end{align}
		
		
		Также мы можем записать:
		%
		\begin{align}
			d = \sqrt{(\mu + k_d)^2 - 4 (c+k_p)}
			\\
			16^2 = 24^2 - 4 (40 + k_p)
			\\
			k_p = 320 / 4 - 40 = 40
		\end{align}
		
		
	\end{flushleft}
\end{frame}




\begin{frame}{Размещение полюсов (Pole-placement)}
	\begin{flushleft}
		
		Метод поиска коэффициентов управления таким образом, чтобы замкнутая система имела желаемые собственные значения, называется \emph{размещением полюсов}.
		
		\bigskip
		
		Как показал предыдущий пример, вручную это делать нелегко. Однако существуют программы, которые находят такие коэффициенты управления автоматически.
		
		\bigskip
		
		В MATLAB есть функция \texttt{K = place(A,B,p)}, где \texttt{p} — желаемые собственные значения матрицы \texttt{(A-B*K)}.
		
		
	\end{flushleft}
\end{frame}










\begin{frame}{Слежение за траекторией, 1}
	\begin{flushleft}
		
		Пусть функция $\bo{x}^* = \bo{x}^*(t)$ и управление $\bo{u}^* = \bo{u}^*(t)$ являются решением системы $\dot{\bo{x}} = \bo{A}\bo{x} + \bo{B}\bo{u}$, то есть:
		%
		\begin{equation}
			\dot{\bo{x}}^* = \bo{A}\bo{x}^* + \bo{B}\bo{u}^*
		\end{equation}
		
		Мы называем $\bo{x}^*(t)$ \emph{задающим воздействием (reference)} или \emph{опорным сигналом}, а $\bo{u}^*(t)$ — \emph{компенсирующим управлением (feed-forward control)}.
		
		\bigskip
		
		Мы можем попытаться найти закон управления, который стабилизирует эту опорную траекторию. Начнём с нахождения разности между $\dot{\bo{x}}^*$ и $\dot{\bo{x}}$:
		%
		\begin{equation}
			\dot{\bo{x}}^* - \dot{\bo{x}}= \bo{A}(\bo{x}^*-\bo{x}) + \bo{B}(\bo{u}^*-\bo{u})
		\end{equation}
		
		Введём новые переменные: $\bo{e} = \bo{x}^* - \bo{x}$ и $\bo{v} = \bo{u}^* - \bo{u}$:
		%
		\begin{equation}
			\dot{\bo{e}} = \bo{A}\bo{e} + \bo{B}\bo{v}
		\end{equation}
		
	\end{flushleft}
\end{frame}



\begin{frame}{Слежение за траекторией, 2}
	%\framesubtitle{Stability of the closed-loop system}
	\begin{flushleft}
		
		Мы называем $\bo{e}$ \emph{ошибкой управления}, а уравнение $\dot{\bo{e}} = \bo{A}\bo{e} + \bo{B}\bo{v}$ — \emph{динамикой ошибки}.
		
		\bigskip
		
		Таким образом, мы вернулись к знакомой задаче: найти закон управления $\bo{v} = -\bo{K}\bo{e}$, который делает замкнутую систему устойчивой:
		%
		\begin{equation}
			\dot{\bo{e}} = (\bo{A} - \bo{B}\bo{K}) \bo{e}
		\end{equation}
		
		В исходных переменных закон управления принимает вид:
		%
		\begin{equation}
			\bo{u} = \bo{K}(\bo{x}^* - \bo{x}) + \bo{u}^*
		\end{equation}
		
	\end{flushleft}
\end{frame}




\begin{frame}{Слежение за траекторией, 3}
	%\framesubtitle{Stability of the closed-loop system}
	\begin{flushleft}
		
		Если для заданного опорного сигнала $\bo{x}^*$ существует компенсирующее управление $\bo{u}^*$, удовлетворяющее $\dot{\bo{x}}^* = \bo{A}\bo{x}^* + \bo{B}\bo{u}^*$, то мы можем легко найти такое $\bo{u}^*$ с помощью псевдообратной матрицы:
		
		\begin{equation}
			\bo{u}^* = \bo{B}^+(\dot{\bo{x}}^*  - \bo{A}\bo{x}^*)
		\end{equation}				
		
		Если желаемая траектория постоянна $\bo{x}^*(t) = \bo{x}_d$ (мы пытаемся переместить систему в новую позицию в пространстве состояний), то компенсирующее управление принимает вид:
		
		\begin{equation}
			\bo{u}^* = -\bo{B}^+\bo{A}\bo{x}_d
		\end{equation}	
		
		
	\end{flushleft}
\end{frame}



\begin{frame}{Новое входное воздействие}
	%\framesubtitle{Stability of the closed-loop system}
	\begin{flushleft}
		
		Рассмотрим систему $\dot{\bo{x}} = \bo{A}\bo{x} + \bo{B}\bo{u}$ и закон управления $\bo{u} = \bo{K}(\bo{x}^*(t) - \bo{x}) + \bo{u}^*(t)$. Мы можем найти выражение для результирующей системы:
		
		\begin{align}
			\dot{\bo{x}} = \bo{A}\bo{x} + \bo{B}\bo{K}(\bo{x}^*(t) - \bo{x}) + \bo{B}\bo{u}^*(t) \\
			\dot{\bo{x}} = (\bo{A}- \bo{B}\bo{K})\bo{x} +\bo{B}\bo{K}\bo{x}^*(t) + \bo{B}\bo{u}^*(t)
		\end{align}		
		
		В предположении, что $\bo{u}^*(t) = 0$, получаем упрощённую систему:
		
		\begin{align}
			\dot{\bo{x}} =  (\bo{A}- \bo{B}\bo{K})\bo{x} +\bo{B}\bo{K}\bo{x}^*(t)
		\end{align}				
		
		Здесь видно, что $\bo{x}^*(t)$ действует как новый вход системы. Мы можем проанализировать реакцию этой системы на входное воздействие.
		
	\end{flushleft}
\end{frame}





\begin{frame}{Литература}
	
	\begin{itemize}
		
		
		\item Nise, N.S. Control systems engineering. John Wiley \& Sons. (4.5 The General Second-Order System)	
		
		\item \bref{https://apmonitor.com/pdc/index.php/Main/ModelSimulation}{Dynamic Simulation in Python}
	\end{itemize}
\end{frame}







\myqrframe

\end{document}
