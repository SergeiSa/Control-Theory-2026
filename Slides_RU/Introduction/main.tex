\documentclass{beamer}

\usepackage[T2A]{fontenc}
\usepackage[utf8]{inputenc}
\usepackage[english,russian]{babel}

\input{settings.tex}


\title{ОДУ и пространство состояний}
\subtitle{Теория управления, лекция 1}
\author{Сергей Савин}
\centering
\date{\mydate}



\begin{document}
\maketitle





\begin{frame}{Роль теории управления}
	%\framesubtitle{}
	\begin{flushleft}
		
		% TODO: \usepackage{graphicx} required
		\begin{figure}
			\centering
			\includegraphics[width=1.00\linewidth]{"Control Examples"}

			\label{fig:control-examples}
		\end{figure}
		
		
	\end{flushleft}
\end{frame}

\begin{frame}{Роль этого курса}
	%\framesubtitle{}
	\begin{flushleft}
		
		Мы изучим:
		
		\begin{itemize}
			\item как анализировать объекты, которыми мы можем управлять;
			
			\item каких целей можно достичь с помощью управления;
			
			\item как осуществлять управление;
			
			\item с какими проблемами мы сталкиваемся при работе с управлением и как их решать.
		\end{itemize}
		
		\bigskip
		
		Это вводный курс, цель которого — подготовить базу для всех продвинутых курсов (Управление автономными транспортными средствами, Нелинейное управление и др.) требующих знания ТАУ.
		
	\end{flushleft}
\end{frame}



\begin{frame}{Модели}
	%\framesubtitle{}
	\begin{flushleft}
		
		Многие физические объекты, которыми мы управляем, могут быть описаны с помощью \emph{ОДУ}.
		
		\bigskip 
		
		Мы начнем этот курс с изучения того, как записывать ОДУ в форме, непосредственно полезной для целей управления, называемой представлением в \emph{пространстве состояний}.
		
	\end{flushleft}
\end{frame}


\begin{frame}{Обыкновенные дифференциальные уравнения 1-го порядка}
	%\framesubtitle{}
	\begin{flushleft}
		
		Вспомним нормальную форму \emph{обыкновенных дифференциальных уравнений (ОДУ) первого порядка:}
		
		\begin{equation}
			\dot{\bo{x}} = \bo{f} (\bo{x}, t)
		\end{equation}
		
		где $\bo{x} = \bo{x}(t)$ — решение уравнения, а $t$ — независимая переменная (обычно — время).
		
		\bigskip
		
		\begin{definition}
			Мы можем называть ОДУ \emph{динамической системой}, а $\bo{x}$ — \emph{состоянием} динамической системы.  
		\end{definition}
		
		\begin{example}
			\begin{equation}
				\dot{x} = -2 x
			\end{equation}
		\end{example}
		
	\end{flushleft}
\end{frame}




\begin{frame}{Состояние}
	%	\framesubtitle{1st order}
	\begin{flushleft}
		
		\emph{Состояние} динамической системы — это минимальный набор переменных, описывающий систему, в том смысле, что зная текущее состояние и все будущие входные воздействия, можно описать будущее поведение системы.
		
		\begin{example}
			Для системы пружина-демпфер переменными состояния могут быть положение и скорость массы.
		\end{example} 
		\begin{example}
			Для двойного маятника переменными состояния могут быть углы в сочленениях и их скорости.
		\end{example} 
		
	\end{flushleft}
\end{frame}






\begin{frame}{ОДУ n-го порядка}
	%\framesubtitle{n-th order}
	\begin{flushleft}
		
		Нормальная форма \emph{обыкновенного дифференциального уравнения n-го порядка} имеет вид:
		
		\begin{equation}
			y^{(n)} = f (y^{(n-1)}, y^{(n-2)}, ...\,, \dot{y}, y, t)
		\end{equation}
		
		где $y = y(t)$ — решение уравнения. Как и ранее, это \emph{динамическая система}, но в этот раз нам нужно больше переменных для описания состояния этой системы, например, мы можем использовать набор $\{ y, \ \dot{y} , ...\,,y^{(n-1)} \}$.
		
		\begin{example}[Маятник]
			\begin{equation}
				\ddot{y} = - 0.1 \dot y - 7\sin(y)
			\end{equation}
		\end{example}
		
		
		\begin{example}[Двигатель постоянного тока при постоянном напряжении]
			\begin{equation}
				\begin{cases}
					\dot{y}_1 = - 100 \dot{y}_2 -2 y_1  + 10 \\
					\ddot{y}_2 = -0.1 \dot{y}_2 + 100 y_1
				\end{cases}
			\end{equation}
		\end{example}
		
	\end{flushleft}
\end{frame}




\begin{frame}{Линейные ОДУ 1-го порядка}
	\begin{flushleft}
		
		Система линейных ОДУ первого порядка может быть записана как:
		
		\begin{equation}
			\dot{\bo{x}} = \bo{A} \bo{x}
		\end{equation}
		
		\begin{example}
			\begin{equation}
				\begin{cases}
					\dot{x}_1 = -20 x_1 + 7 x_2 \\
					\dot{x}_2 = 10 x_1 - 3 x_2
				\end{cases}
			\end{equation}
		\end{example}
		
		\begin{example}
			\begin{equation}
				\begin{bmatrix}
					\dot{x}_1 \\
					\dot{x}_2
				\end{bmatrix} 
				= 
				\begin{bmatrix}
					-20   & 7  \\
					10 & -3
				\end{bmatrix}
				\begin{bmatrix}
					x_1 \\
					x_2
				\end{bmatrix} 
			\end{equation}
		\end{example}
		
	\end{flushleft}
\end{frame}




\begin{frame}{Линейные дифференциальные уравнения n-го порядка}
	%\framesubtitle{n-th order}
	\begin{flushleft}
		
		Одно линейное ОДУ n-го порядка часто записывают в форме:
		
		\begin{equation}
			a_n y^{(n)} + 
			... +
			a_2 \ddot{y} + a_1 \dot{y} + 
			a_0 y = 0
		\end{equation}
		
		\begin{example}
			\begin{equation}
				12 \dddot{y}  + 5 \dot{y} + 
				2 y = 0
			\end{equation}
		\end{example}
		
		\begin{example}
			\begin{equation}
				5 \ddot{y} - \dot{y} + 
				10 y = 0
			\end{equation}
		\end{example}
		
	\end{flushleft}
\end{frame}


\begin{frame}{ОДУ с входным воздействием, часть 1}
	%\framesubtitle{n-th order}
	\begin{flushleft}
		
		Иногда удобно записывать ОДУ в форме с \emph{входным воздействием}, например:
		
		\begin{equation}
			a_2 \ddot{y} + a_1 \dot{y} + 
			a_0 y = u(t)
		\end{equation}
		
		В этом уравнении $u(t)$ — функция времени. Такая форма предлагает нам много возможностей:
		
		\begin{itemize}
			\item Мы можем использовать $u(t)$ для моделирования \emph{управляющего воздействия} (например, напряжения, момента двигателя), которым мы непосредственно управляем.
			
			\item Мы можем использовать $u(t)$ для моделирования внешних сил, действующих на систему.
			
			\item Мы можем подставить конкретную функцию вместо $u(t)$, например, синусоиду или ступенчатую функцию, чтобы изучить поведение системы при таком входном воздействии.
		\end{itemize}
		
	\end{flushleft}
\end{frame}



\begin{frame}{ОДУ первого порядка с входным воздействием}
	\begin{flushleft}
		
		Некоторые примеры линейных ОДУ с одним входным воздействием:
		
		
		\begin{example}
			\begin{equation}
				\begin{cases}
					\dot{y}_1 = -20 y_1 + 7 y_2 + u \\
					\dot{y}_2 = 10 y_1 - y_2
				\end{cases}
			\end{equation}
		\end{example}
		
		\begin{example}
			\begin{equation}
				\begin{bmatrix}
					\dot{x}_1 \\
					\dot{x}_2
				\end{bmatrix} 
				= 
				\begin{bmatrix}
					-20  & 7 \\
					10 & -1
				\end{bmatrix}
				\begin{bmatrix}
					x_1 \\
					x_2 
				\end{bmatrix} 
				+
				\begin{bmatrix}
					1    \\
					0  
				\end{bmatrix}
				u
			\end{equation}
		\end{example}
		
	\end{flushleft}
\end{frame}



\begin{frame}{ОДУ и пространство состояний}
	%\framesubtitle{n-th order}
	\begin{flushleft}
		
		Общая форма линейного ОДУ n-го порядка с входным воздействием может быть представлена следующим образом:
		%
		\begin{equation}
			\label{eq:ODE_1}
			a_n y^{(n)} + 
			... +
			a_2 \ddot{y} + a_1 \dot{y} + 
			a_0 y = u(t)
		\end{equation}
		
		\bigskip
		
		Представление в пространстве состояний линейной системы с входным воздействием имеет вид:
		%
		\begin{equation}
			\label{eq:SS_1}
			\dot{\bo{x}} = \bo{A} \bo{x} + \bo{B} \bo{u}
		\end{equation}
		
		Обратите внимание, что в \eqref{eq:ODE_1} $u$ — скаляр, тогда как в \eqref{eq:SS_1} $\bo{u}$ может быть как скаляром, так и вектором.
		
	\end{flushleft}
\end{frame}







\begin{frame}{Уравнения с выходом}
	%\framesubtitle{n-th order}
	\begin{flushleft}
		
		Уравнения также могут иметь выход. Что означает "выход" — зависит от конкретного случая применения; это не математический вопрос, а вопрос интерпретации. Например, выход может означать:
		
		\begin{itemize}
			\item То, что мы измеряем (положение и ориентацию квадрокоптера, угловую скорость ротора двигателя и т.д.).
			
			\item То, что нас интересует и/или что мы хотим контролировать (высоту квадрокоптера, скорость автомобиля и т.д.)
			
			\item и другое.
		\end{itemize}
		
		Мы часто обозначаем выход как $y$, и он зависит от состояния системы: $y = g(\bo{x})$
		
	\end{flushleft}
\end{frame}


\begin{frame}{Уравнения с выходом}
	%\framesubtitle{n-th order}
	\begin{flushleft}
		
		Представление в пространстве состояний линейной системы с входным воздействием и выходом имеет вид:
		%
		\begin{equation}
			\begin{cases}
				\dot{\bo{x}} = \bo{A} \bo{x} + \bo{B}\bo{u} \\
				\bo{y} = \bo{C}\bo{x}
			\end{cases}
		\end{equation}
		
		Если $\bo{u} \in \R$ и $\bo{y} \in \R$ (т.е. если они являются скалярами), и вы хотите представить систему с выходом в виде одного ОДУ, обычно рассматривают выход как переменную ОДУ:
		
		\begin{equation}
			a_n y^{(n)} + 
			... +
			a_2 \ddot{y} + a_1 \dot{y} + 
			a_0 y = u(t)
		\end{equation}
		
		
		Если $\bo{u}$ и $\bo{y}$ являются скалярами, система называется \emph{одновходовой-одновыходной (SISO)}, если они являются векторами — \emph{многовходовой-многовыходной (MIMO)}.
		
		\bigskip
		
		Мы всегда можем выразить SISO-систему в любой из форм — ОДУ или пространство состояний.
		
	\end{flushleft}
\end{frame}





\begin{frame}{Преобразование ОДУ в пространство состояний}
	% \framesubtitle{...are what we will study}
	\begin{flushleft}
		
		Рассмотрим уравнение $\dddot{y} + a_2 \ddot{y} + a_1 \dot{y} + a_0 y =u$.
		
		\bigskip
		
		Выполним подстановку: $x_1 = y$, $x_2 = \dot{y}$, $x_3 = \ddot{y}$. Получим:
		
		\begin{align}
			\dot{x}_1 &= \dot{y} = x_2 \\
			\dot{x}_2 &= \ddot{y} = x_3 \\
			\dot{x}_3 &=  \dddot{y} = 
			u-a_2 \ddot{y} - a_1 \dot{y} - a_0 y = 
			u-a_2 x_3 - a_1 x_2 - a_0 x_1
		\end{align}
		
		Что можно непосредственно записать в форме пространства состояний:
		
		\begin{equation}
			\begin{bmatrix}
				\dot{x}_1 \\ \dot{x}_2 \\ \dot{x}_3
			\end{bmatrix} 
			=
			\begin{bmatrix}
				0 & 1 & 0 \\ 
				0 & 0 & 1 \\
				-a_0 & -a_1 & -a_2
			\end{bmatrix} 
			\begin{bmatrix}
				x_1 \\ x_2 \\ x_3
			\end{bmatrix} 
			+ 
			\begin{bmatrix}
				0 \\ 0 \\ u
			\end{bmatrix}
		\end{equation}
		
		
	\end{flushleft}
\end{frame}






\begin{frame}{Управляемая каноническая форма}
	% \framesubtitle{...are what we will study}
	\begin{flushleft}
		
		В общем случае следующая форма (результат преобразования ОДУ в пространство состояний) называется управляемой канонической формой:
		
		\begin{align}
			\begin{cases}
			\dot{\bo{x}}
			&=
			\begin{bmatrix}
				0             & 1              & 0   & ... & 0 \\ 
				0             & 0             & 1    & ... & 0 \\
				...            & ...             & ...  & ... & ...  \\
				-a_{n-1} & -a_{n-2} & -a_{n-3} & ...  & -a_0
			\end{bmatrix} 
			\bo{x}
			+ 
			\begin{bmatrix}
				0 \\ 0\\ ... \\ 1
			\end{bmatrix}
			u
			\\
			y &= 			
			\begin{bmatrix}
				c_{n-1} & c_{n-1} &   ... & c_0
			\end{bmatrix}
			\bo{x}
		\end{cases}
		\end{align}
		
		
	\end{flushleft}
\end{frame}



\begin{frame}{Литература}

\begin{itemize}
	
	
	\item K. Ogata Modern Control Engineering (Chapter 2.4)
	
\item Nise, N.S. Control systems engineering. John Wiley \& Sons. (Chapter 3 Modeling in Time Domain)	
	
\item Dorf \& Bishop, Modern Control Systems (Chapter 3.3 State differential equation)		
	
\item 2.14 Analysis and Design of Feedback Control Systems:

\begin{itemize}
	\item  \bref{http://web.mit.edu/2.14/www/Handouts/StateSpace.pdf}{State-Space Representation of LTI Systems}
	
	\item  \bref{http://web.mit.edu/2.14/www/Handouts/StateSpaceResponse.pdf}{Time-Domain Solution of LTI State Equations}
\end{itemize}	
	
\item \bref{https://lpsa.swarthmore.edu/}{Linear Physical Systems Analysis}:

\begin{itemize}
\item State Space Representations of Linear Physical Systems \bref{https://lpsa.swarthmore.edu/Representations/SysRepSS.html}{lpsa.swarthmore.edu/Representations/SysRepSS.html}

\item Transformation: Differential Equation to State Space \bref{https://lpsa.swarthmore.edu/Representations/SysRepTransformations/DE2SS.html}{lpsa.swarthmore.edu/.../DE2SS.html}
\end{itemize}	

\end{itemize}

\end{frame}



\myqrframe



\begin{frame}{State Space to ODE}
	%\framesubtitle{part 5}
	\begin{flushleft}
		
		\textcolor{blue}{\href{https://github.com/SergeiSa/Control-Theory-Slides-Spring-2022/blob/main/ColabNotebooks/StateSpace2ODE.ipynb}{Check out the code implementation.}}
		
		\bigskip
		
		
		\centerline{\textcolor{black}{\qrcode[height=2.1in]{https://github.com/SergeiSa/Control-Theory-Slides-Spring-2022/blob/main/ColabNotebooks/StateSpace2ODE.ipynb}}}
		
		
	\end{flushleft}
\end{frame}



\begin{frame}{Приложение}
	\framesubtitle{Преобразование из пространства состояний в ОДУ}
	\begin{flushleft}
		
		Мы можем далее обобщить ОДУ до следующей формы:
		%
		\begin{equation}
			a_n y^{(n)} + 
			... +
			a_2 \ddot{y} + a_1 \dot{y} + 
			a_0 y = h_{n-1} u^{(n-1)} + ... + h_0 u
		\end{equation}
		
		Мы можем представить любую\footnotemark систему SISO в пространстве состояний в такой форме. 
		
		\bigskip
		
		Сначала дифференцируем $y = \bo{C}\bo{x}$ $n$ раз:
		%
		\begin{align}
			\dot{y} &= \bo{C}\dot{\bo{x}} = \bo{C}\bo{A} \bo{x} + \bo{C}\bo{B}u
			\\
			\ddot{y} &= \bo{C}\bo{A} \dot{\bo{x}} + \bo{C}\bo{B}\dot{u}  = 
			\bo{C}\bo{A}^2 \bo{x} + \bo{C}\bo{A}\bo{B}u + \bo{C}\bo{B}\dot{u} 
			\\
			&	...
			\\
			y^{(n)} &= \bo{C}\bo{A}^n \bo{x} + \bo{C}\bo{A}^{n-1}\bo{B}u + ... + \bo{C}\bo{B}u^{(n-1)}
		\end{align}
		
		
		
		\footnotetext[1]{система должна быть наблюдаемой.} 
		
	\end{flushleft}
\end{frame}





\begin{frame}{Приложение}
	\framesubtitle{Преобразование из пространства состояний в ОДУ}
	\begin{flushleft}
		
		
		Далее записываем характеристическое уравнение матрицы $\bo{A}$:
		%
		\begin{align}
			\text{det}(s\bo{I} - \bo{A}) = 0
			\\
			p(s) = s^n + \alpha_{n-1} s^{n-1} + ... +  \alpha_0 = 0
		\end{align}
		
		Согласно теореме Кэли-Гамильтона, $p(\bo{A}) = \bo{0}$. Следовательно:
		%
		\begin{align}
			p(\bo{A}) &= \bo{A}^n + \alpha_{n-1} \bo{A}^{n-1} + ... +  \bo{I}\alpha_0 =  \bo{0}
		\end{align}		
		
		Таким образом, мы можем выразить $\bo{A}^n$ как функцию младших степеней:
		%
		\begin{align}
			\bo{A}^n = -\alpha_{n-1} \bo{A}^{n-1} - ... -  \bo{I}\alpha_0 
		\end{align}		
		
		
	\end{flushleft}
\end{frame}




\begin{frame}{Приложение}
	\framesubtitle{Преобразование из пространства состояний в ОДУ}
	\begin{flushleft}
		
		
		Ранее мы нашли член $ \bo{C}\bo{A}^n \bo{x}$ в выражении для $y^{(n)}$ — единственный член, содержащий переменную состояния $\bo{x}$. Раскроем его, используя теорему Кэли-Гамильтона:
		%
		\begin{align}
			\bo{C}\bo{A}^n \bo{x} = -\alpha_{n-1}  \bo{C}\bo{A}^{n-1} \bo{x} - ... -  \alpha_0  \bo{C}\bo{x}
		\end{align}		
		
		Но $\bo{C}\bo{x} = y$, ... , $\bo{C}\bo{A} \bo{x} = \dot{y} -  \bo{C}\bo{B}u$, и т.д. Подставляя это в выражение для $y^{(n)}$, находим ОДУ только через входно-выходные переменные (переменные состояния были исключены):
		%
		\begin{align}
			y^{(n)} &= m_{n-1} y^{(n-1)} + ... + m_0 y + p_{n-1} u^{(n-1)} + ... + p_0 u
		\end{align}		
		
	\end{flushleft}
\end{frame}


\end{document}
